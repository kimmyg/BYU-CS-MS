\chapter{CPS transformation}

\section{Introduction}

The CPS transform is actually a family of language transformations derived from Plotkin
\cite{plotkin1975call} designed to simplify  programs by representing all data and control
flow uniformly and explicitly. This, in turn, simplifies compiler construction and
analyses such as optimization and verification \cite{appel2007compiling, sabry1994formal}. The standard
variation of CPS adds a formal parameter to every function definition and an argument to
every call site.

As an example, consider once again the two variants of the factorial function, sans
continuation marks, given earlier. In CPS, the properly-recursive variant can be expressed
as
\begin{schemeblock}
\begin{schemedisplay}
(define (fac n k) 
  (if (= n 0)
      (k 1)
      (fac (- n 1) (lambda (acc) (k (* n acc))))))
\end{schemedisplay}
\end{schemeblock}
\noindent
and the tail-recursive variant as
\begin{schemeblock}
\begin{schemedisplay}
(define (fac-tr n acc k)
  (if (= n 0)
      (k acc)
      (fac-tr (- n 1) (* n acc) k)))
\end{schemedisplay}
\end{schemeblock}
\noindent
(For clarity, we have treated ``primitive'' functions--equality comparison, subtraction,
and multiplication--in a direct manner. In contrast, a comprehensive CPS transformation
would affect \emph{every} function.)

Notice that, in the first variation, each recursive call receives a newly-constructed
\scheme{k} encapsulating additional work to be performed at the completion of the
recursive computation. In the second, \scheme{k} is passed unmodified, so while
computation occurs within each context, no \emph{additional} computation pends. From this
example, we see that the CPS representation is ideal for understanding tail-call behavior
as it is explicit that the continuation is preserved by the tail call.

The purpose of CPS does not lie solely in pedagogy, however. The reification of and
consequent ability to directly manipulate the continuation is powerful, analogous in power
to the ability to \emph{capture} a continuation which some languages provide. In Scheme,
this is accomplished with \scheme{call/cc}, short for ``call with current continuation''.
This call takes one argument which itself is a function of one argument. \scheme{call/cc}
calls its argument, passing in a functional representation of the current
continuation--the continuation present when \scheme{call/cc} was invoked. This
continuation function takes one argument which is treated as the result of
\scheme{call/cc} and runs this continuation to completion.

As a simple example,
\begin{schemeblock}
\begin{schemedisplay}
(+ 1 (call/cc
       (lambda (k)
         (k 1))))
\end{schemedisplay}
\end{schemeblock}
returns \scheme'2'. In effect, invoking \scheme{k} with \scheme'1' is the same 
as replacing the entire \scheme{call/cc} invocation with \scheme'1'.

Much of the power of \scheme{call/cc} lies in the manifestation of the continuation as a
function, giving it first-class status. It can be passed as an argument in function calls,
invoked, and, amazingly, reinvoked at leisure. It is this reinvokeability that makes
\scheme{call/cc} the fundamental unit of control from which all other control structures can
be built, including generators, coroutines, and threads.

%\emph{call/cc} is erroneously seen as incredibly heavyweight and overkill for control
%(cite something). This conception probably comes from the conceptualization of the
%continuation as the call stack and continuation capture as stack copy while continuation
%call is stack installation. It is also seen as a form of \emph{goto} which is known to
%obfuscate control flow and impede analysis (cite something). [Can't and shouldn't talk
%much about the second. Probably take it out.] Bearing in mind that the CPS transformation
%aids compilers, it is useful to investigate the characterization of \emph{call/cc} within
%the standard CPS transformation.

In direct style, the definition of \scheme{call/cc} is conceptually 
\begin{schemeblock}
\begin{schemedisplay}
(define call/cc
  (lambda (f)
    (f (get-function-representing-continuation))))
\end{schemedisplay}
\end{schemeblock}
where \scheme{get-function-representing-continuation} is an opaque function which leverages
sweeping knowledge of the language implementation. The CPS definition is notably simpler:
\begin{schemeblock}
\begin{schemedisplay}
(define call/cc
  (lambda (f k)
    (f (lambda (x w) (k x)) k)))
\end{schemedisplay}
\end{schemeblock}
Within the definition of \scheme'call/cc', we define an anonymous function which, when given a value $x$ and continuation $w$, applies the captured continuation $k$ to $x$.

Variations of the standard CPS transformation make the expression of certain control
structures more straightforward. For instance, the ``double-barrelled'' CPS transformation
is a variation wherein each function signature receives not one but two additional formal
parameters, each a continuation. One application of this particular variation is error
handling with one continuation argument representing the remainder of a successful
computation and the other representing the failure contingency. It is especially useful in
modelling exceptions and other non-local transfers of control in situations where the
computation might fail. In general, the nature of the CPS transformation allows it to
untangle complicated, intricate control structures.

Similar transformations exist which express other programming language features such as
security annotations \cite{wallach2000safkasi} and control structures such as procedures,
exceptions, labelled jumps, coroutines, and backtracking. On top of other offerings, this
places it in a category of tools to describe and analyze programming language features.
(This category is also occupied by Moggi's computational \lc--monads
\cite{moggi1989computational}.)

\section{Example}

We will now focus our attention on a CPS transform defined over \lv.

A CPS transformation is a global syntactic transformation of language terms. Recall that
terms in the \lc\ take the form of lone variables $x$, $\lambda$-abstractions
$\abs{x}{M}$, and applications $\app{M}{N}$ where $M$ and $N$ are themselves \lc\ terms. A
comprehensive CPS transform definition then need only specify transformations for these
three categories. As an example, consider Fischer's CPS transform \cite{fischer1972lambda}:
\begin{align*}
\mathcal{F}[x]          &= \abs{k}{\app{k}{x}}\\
\mathcal{F}[\abs{x}{M}] &= \abs{k}{\app{k}{\abs{x}{\mathcal{F}[M]}}}\\
\mathcal{F}[\app{M}{N}] &= \abs{k}{\app{\mathcal{F}[M]}{\abs{m}{\app{\mathcal{F}[N]}{\abs{n}{\app{\app{m}{n}}{k}}}}}}
\end{align*}
Fischer's CPS transform abstracts each term in the \lc: lone variables wait on a
continuation, abstractions receive a degree of indirection, and even applications, the
sole reduction facility of the \lc, become abstractions. In essence, terms become
suspended in wait of a continuation argument. By priming a term so-transformed with a
continuation function--even as simple as the identity function--we instigate a cascade of
computation.

In a sense, the CPS transform contaminates abstractions, the values of the \lc. For example, consider the
transformation of $\lambda x.x$
\begin{align*}
\mathcal{F}[\lambda x.x] &= \lambda k.k\,(\lambda x.\mathcal{F}[x])\\
                         &= \lambda k.(k\,\lambda x.\lambda k.(k\,x))
\end{align*}
If we apply this term to the identity function, it reduces as
\begin{align*}
                        &\lambda k.(k \lambda x.\lambda k.(k\,x))\,\lambda y.y\\
\lvrr &\lambda y.y\,\lambda x.\lambda k.(k\,x)\\
\lvrr &\lambda x.\lambda k.(k\,x)
\end{align*}
The result of this reduction contains residue of the CPS transform. This must be accounted for when attempting to formally relate direct and continuation-passing style terms.

