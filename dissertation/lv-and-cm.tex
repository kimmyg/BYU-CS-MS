\chapter{\lv\ and \cm}

\section{\lv}

The \lv\ language is merely the call-by-value \lc\ \cite{plotkin1975call}.

Figure \ref{lv-language-forms} presents the language terms and evaluation contexts of \lv.
The definition of $e$ specifies the form of terms in \lv\ similar to our previous
discussion of the \lc. The definition of $E$ specifies evaluation contexts which, like
terms, are defined inductively. An evaluation context is either empty (denoted by
$\hole$), an application wherein the operator is being evaluated, or an application
wherein the operand is being evaluated.

\begin{figure}
\begin{align*}
E = &\app{E}{e} & e = &\app{e}{e}\\
    &\app{v}{E} &     &x\\
    &\hole      &     &v\\
    &           & v = &\abs{x}{e}
\end{align*}
\caption{\lv\ forms}
\label{lv-language-forms}
\end{figure}

Figure \ref{lv-language-semantics} presents the sole semantic definition of \lv, the
meaning of application.

\begin{figure}
\begin{align*}
E[(\lambda x.e)\,v]     &\rightarrow E[e[x\leftarrow v]]\\
\end{align*}
\caption{\lv\ evaluation rule}
\label{lv-language-semantics}
\end{figure}

\section{\cm}

We now consider an extension of \lv\ with facilities to manipulate continuation 
marks, introduced by Pettyjohn et al. \cite{pettyjohn2005continuations}, which 
we term \cm. As an extension of \lv, it inherits its definitions of language 
terms, evaluation contexts, and semantics.

Figure \ref{cm-language-forms} presents the syntactic forms of \cm.  Definitions of $E$
and $F$ signify evaluation contexts. The separation of $E$ from $F$ prevents $\wcm{v}{F}$
contexts from directly nesting within the entire evaluation context to enforce proper
tail-call behavior. The definition of $e$ is identical to that of \lv\ with the addition
of contionuation mark forms $\wcm{e}{e}$ and $\ccm$. (\cm\ expresses unkeyed marks which
obviates the need to specify a key to which a value will be associated. Hence, the
\scheme'wcm' and \scheme'ccm' forms need one parameter fewer than their Scheme
counterparts.)

\begin{figure}
\begin{align*}
E = &\wcm{v}{F} & e = &\app{e}{e}\\
    &F          &     &x\\
F = &\hole      &     &v\\
    &\app{E}{e} &     &\wcm{e}{e}\\
    &\app{v}{E} &     &\ccm\\
    &\wcm{E}{e} & v = &\abs{x}{e}
\end{align*}
\caption{\cm\ forms}
\label{cm-language-forms}
\end{figure}

Figure \ref{cm-language-semantics} presents the semantics of \cm\ in the form of a list of
reduction rules. Rule 1 defines the meaning of application as inherited from \lv. Rule 2
defines the tail behavior of the \scheme'wcm' form. Rule 3 expresses that the \scheme'wcm'
form takes on the value of its body. Finally, rule 4 defines the value of the \scheme'ccm'
form in terms of the $\chi$ metafunction.

\begin{figure}
\begin{align*}
E[(\lambda x.e)\,v]     &\rightarrow E[e[x\leftarrow v]]\tag{1}\\
E[\wcm{v}{\wcm{v'}{e}}] &\rightarrow E[\wcm{v'}{e}]\tag{2}\\
E[\wcm{v}{v'}]          &\rightarrow E[v']\tag{3}\\
E[\ccm]                 &\rightarrow E[\chi(E)]\tag{4}
\end{align*}
\caption{\cm\ semantics}
\label{cm-language-semantics}
\end{figure}

The definition of the $\chi$-metafunction is given in figure \ref{cm-chi-metafunction}.
Conceptually, the $\chi$-metafunction traverses the context from the outside in,
accumulating values as it encounters $\wcm{v}{E}$ contexts. Since the $\chi$ metafunction
is defined over evaluation contexts of \cm, its domain corresponds to the definitions of
$E$ and $F$ in figure \ref{cm-language-forms}.

\begin{figure}
\begin{align*}
\chi(\hole)             &= \mathrm{nil}\\
\chi(E[\app{\hole}{e}]) &= \chi(E)\\
\chi(E[\app{v}{\hole}]) &= \chi(E)\\
\chi(E[\wcm{\hole}{e}]) &= \chi(E)\\
\chi(E[\wcm{v}{\hole}]) &= v : \chi(E)
\end{align*}
\caption{Definition of $\chi$ metafunction}
\label{cm-chi-metafunction}
\end{figure}

% what are the contracts? why does it work?

% most states have a direct correspondence in C
% (E,e) being a state
% most behavior has a direct correspondence in C (evaluation steps)
% reinsertion of value, composition of contexts, semantic rules



In \cm, evaluation of an application form proceeds as

\begin{align*}
E[\app{e_0}{e_1}]\cmrr &E[\app{\hole}{e_1}][e_0]\\
                \cmrrs &E[\app{\hole}{e_1}][v_0]\\
                 \cmrr &E[\app{v_0}{e_1}]\tag{1}\\
                 \cmrr &E[\app{v_0}{\hole}][e_1]\\
                \cmrrs &E[\app{v_0}{\hole}][v_1]\\
                 \cmrr &E[\app{v_0}{v_1}]\tag{2}\\
                     = &E[\app{\abs{x}{e'_0}}{v_1}] &\text{for some $x$ and $e'_0$}\\
                 \cmrr &E[e'_0[x\leftarrow v_1]]\\
                     = &E[e_2] &\text{for some $e_2$}
\end{align*}

Steps 1 and 2 denote the insertion of a value into the hole in the context.

