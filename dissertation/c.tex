\chapter{$\mathcal{C}$}

A CPS-like global transformation can compile the \lc\ with continuation marks into
the plain \lc\ in a semantics-preserving way.

In order to demonstrate this, we must first clarify (and later, formalize) the semantics-preservation property. We will define a transformation from \cm\ to \lv\ and term it $\mathcal{C}$, as in \emph{compile}, since we are, in essence, compiling away continuation marks. In order to preserve the meaning of \cm, $\mathcal{C}$ must commute with evaluation. More precisely, $\mathcal{C}$ satisfies
\[
\begin{array}{ccc}
p & \cmrrs & v\\
\downarrow_\mathcal{C} & & \downarrow_\mathcal{C}\\
\C{p} & \lvrrs & \C{v}
\end{array}
\]
for any program $p\in\lambda_{cm}$.

The burden of demonstration is then reduced to the existence of $\mathcal{C}$. We do this by construction.

\section{Intuition}

The essence of \cm\ is that programs can apply information to and observe information about the context in which they are evaluated. Programs in \lv\ have no such facility. We can simulate this facility by explicitly passing contextual information to each term as it is evaluated. We can define $\mathcal{C}$ to transform \scheme'wcm' directives to manipulate this information and \scheme'ccm' directives to access it. Intuitively, we can transform \cm\ programs to mark-passing style.

However, marks alone do not account for the tail-call behavior specified by rule 2 of figure \ref{cm-language-semantics}. Since tail-call behavior is observable (if indirectly) by \cm\ programs, we must also provide to each term information about the position in which it is evaluated. Specifically, each transformed \scheme'wcm' directive must be notified whether it is evaluated in tail position of an enclosing \scheme'wcm' directive as it must behave specially if so. Thus, in addition to passing the current continuation marks, the transform should pass a flag to each term indicating whether it is evaluated in tail position of a \scheme'wcm' directive.

%Since this is a dynamic property of the evaluation, we do not attempt to infer it statically.

These two pieces of information suffice to correctly simulate continuation marks.

\section{Concept}

The definition of $\mathcal{C}$ entails transformation over each syntactic form of \cm.

With this in mind, consider a conceptual transformation of application, \scheme|C[(rator-expr rand-expr)]|, as
\begin{schemeblock}
\begin{schemedisplay}
(lambda (flag)
  (lambda (marks)
    (let ((rator-value ((C[rator-expr] FALSE) marks))
          (rand-value ((C[rand-expr] FALSE) marks))
      (((rator-value rand-value) flag) marks))))
\end{schemedisplay}
\end{schemeblock}
ignoring for the moment that \scheme'let' is in neither \lv\ or \cm. This definition captures that
\begin{enumerate}
\item before evaluation, we expect \scheme'flag' to indicate tail position information and \scheme'marks' to provide a list of the current continuation marks,
\item we would like to evaluate \scheme'C[rator-expr]' and \scheme'C[rand-expr]' in the same manner, providing to each its contextual information--specifically that neither is evaluated in tail position of a \scheme'wcm' directive and the continuation marks for each are unchanged from the parent context, and
\item following evaluation of operator and operand and application, evaluation of the resultant term is performed with the original contextual information.
\end{enumerate}

Now consider a conceptual transformation of a \scheme'wcm' directive, \scheme|C[(wcm mark-expr body-expr)]|, as
\begin{schemeblock}
\begin{schemedisplay}
(lambda (flag)
  (lambda (marks)
    ((C[body-expr] TRUE) (let ((mark-value ((C[mark-expr] FALSE) marks))
                               (rest-marks (if flag (snd marks) marks)))
                           (cons mark-value rest-marks)))))
\end{schemedisplay}
\end{schemeblock}
with similar caveats as the previous case. This definition captures that
\begin{enumerate}
\item as in application, we expect \scheme'flag' to indicate tail position information and \scheme'marks' to provide a list of the current continuation marks,
\item we evaluate \scheme'mark-expr' with correct contextual information,
\item we discard the first continuation mark of the parent context if evaluation is occurring in tail position of a \scheme'wcm' directive, and
\item we evaluate \scheme'C[body-expr]' with the correct tail-position flag and current continuation marks.
\end{enumerate}

Finally, consider the conceptual transformation of a \scheme'ccm' directive, \scheme|C[(ccm)]|, as
\begin{schemeblock}
\begin{schemedisplay}
(lambda (flag)
  (lambda (marks)
    marks))
\end{schemedisplay}
\end{schemeblock}
wherein we reap the fruits of simplicity from our laborious passing: this definition is gratifyingly direct.

The conceptual transformation of variables \scheme'x' and values \scheme'(lambda (x) e)' is straightforward.

We now address the absence of \scheme'let', \scheme'if', \scheme'cons', etc. from \lv.

We can express the \scheme'let' construct in \lv\ with application. An expression such as
\begin{schemeblock}
\begin{schemedisplay}
(let ((x_1 e_1)
      ...
      (x_n e_n))
  e)
\end{schemedisplay}
\end{schemeblock}
can be interpreted as
\begin{schemeblock}
\begin{schemedisplay}
(...((lambda (x_1) ... (lambda (x_n)
           e)...) e_1) ...) e_n)
\end{schemedisplay}
\end{schemeblock}
which is the curried form of
\begin{schemeblock}
\begin{schemedisplay}
((lambda (x_1 ... x_n)
           e) e_1 ... e_n)
\end{schemedisplay}
\end{schemeblock}
We unfold this characterization of \scheme'let' to guide the construction of $\mathcal{C}$ before simplifying.

To achieve \scheme'if' and conditionals as well as list primitives \scheme'cons', \scheme'snd', and \scheme'nil', we use the Church encodings of fig. \ref{church-encodings}.


\begin{figure}
\label{church-encodings}
\caption{Church encodings for booleans and lists.}

\begin{definition}
$\true=\abs{x}{\abs{y}{x}}$
\end{definition}

\begin{definition}
$\false=\abs{x}{\abs{y}{y}}$
\end{definition}

\begin{definition}
$\mathbf{cons}=\abs{a}{\abs{b}{\abs{z}{\app{\app{z}{a}}{b}}}}$
\end{definition}

\begin{definition}
$\mathbf{fst}=\abs{p}{\app{p}{\true}}$
\end{definition}

\begin{definition}
$\mathbf{snd}=\abs{p}{\app{p}{\false}}$
\end{definition}

\begin{definition}
$\nil=\false$
\end{definition}
\end{figure}

\section{Initiation}

Abstracting terms has the effect of suspending evaluation. When an entire program is transformed, all evaluation is suspended, and awaits arguments representing contextual information. At the top level, the context is empty, so we pass the contextual information for the empty context: \scheme'FALSE', indicating evaluation is \emph{not} occurring in \scheme'wcm' tail position and \scheme'NIL', an empty list of marks.

We can accommodate this by defining a top-level transform $\hat{\mathcal{C}}$ in terms of $\mathcal{C}$ by
\begin{equation}
\Ch{p}=\app{\app{\C{p}}{\false}}{\nil}
\end{equation}

and stating our commutativity property as
\begin{equation}
\label{commutativity-property}
\Ch{\evalcm{p}}=\evalv{\Ch{p}}
\end{equation}
which is equivalent to
\begin{equation}
\app{\app{\C{\evalcm{p}}}{\false}}{\nil}=\evalv{\app{\app{\C{p}}{\false}}{\nil}}
\end{equation}

\section{Some Final Subtleties}

Our choice to keep the core language small by omitting lists as primitive values has the consequence of complicating our transform somewhat. Because lists are defined in terms of \lc\ values which are themselves touched by the transform and because of the commutativity property that $\mathcal{C}$ must satisfy, we cannot deal with a list of continuation marks directly--we must instead deal with a transformed list of transformed continuation marks, and manipulation of this list within transformed terms must occur at the transformed level.

Additionally, after evaluation, values are ``truncated'' with their leading abstractions applied away. For instance, the transformation of the value \scheme|(lambda (x) x)| to \scheme|(lambda (flag) (lambda (marks) (lambda (x) (lambda (flag) (lambda (marks) x)))))| will yield, following evaluation, \scheme|(lambda (x) (lambda (flag) (lambda (marks) x)))|. For convenience, we define 
\begin{equation}
\Cp{\abs{x}{e}}=\abs{x}{\C{e}}
\end{equation}
and we adjust $\hat{\mathcal{C}}$ so that
\begin{equation}
\Ch{p}=\app{\app{\C{p}}{\false}}{\Cp{\nil}}
\end{equation}

\section{Definition of $\mathcal{C}$}

Finally, we present the definition of $\mathcal{C}$ over the five syntactic forms of \cm.

\begin{schemedefinition}{\scheme|C[(rator-expr rand-expr)]|}
\noindent
The formal transformation of application follows the \scheme'let' version exactly except the definitions of \scheme'rator-value' and \scheme'rand-value' are folded directly in.
\begin{schemeblock}
\begin{schemedisplay}
(lambda (flags)
  (lambda (marks)
    (((((C[rator-expr] FALSE) marks)
       ((C[rand-expr] FALSE) marks))
      flags)
     marks)))
\end{schemedisplay}
\end{schemeblock}
\end{schemedefinition}

\begin{schemedefinition}{\scheme|C[(wcm mark-expr body-expr)]|}
\noindent
The formal transformation of a \scheme'wcm' directive is also extremely similar to the \scheme'let' version. The definition of \scheme'C[cons]' is unfolded and simplified.
\begin{schemeblock}
\begin{schemedisplay}
(lambda (flag)
  (lambda (marks)
    ((C[body-expr] TRUE)
     (((lambda (mark-value) (lambda (rest-marks) Chcps[((CONS mark-value) rest-marks)]))
       ((C[mark-expr] FALSE) marks))
      ((flag Ch[(SND marks)]) marks)))))
\end{schemedisplay}
\end{schemeblock}
\end{schemedefinition}

\begin{schemedefinition}{\scheme|C[(ccm)]|}
\noindent
The \scheme'let' version of the transformation of a \scheme'ccm' directive remains unchanged.
\begin{schemeblock}
\begin{schemedisplay}
(lambda (flag)
  (lambda (marks)
    marks))
\end{schemedisplay}
\end{schemeblock}
\end{schemedefinition}

\begin{schemedefinition}{\scheme|C[v]|=\scheme|C[(lambda (x) e)]|}
\noindent
Like other terms, values are modified to receive contextual information. However, being unaffected by context, values discard this information.
\begin{schemeblock}
\begin{schemedisplay}
(lambda (flag)
  (lambda (marks)
    (lambda (x) C[e])))
\end{schemedisplay}
\end{schemeblock}
\end{schemedefinition}

\begin{schemedefinition}{\scheme|C[x]|}
\noindent
Variables have the property that, when substitution occurs, they reconstitute transformed values. That is, in the midst of application in $\mathcal{C}$, terms of the form \scheme|(Cp[(lambda (x) x)] Cp[(lambda (y) y)])| appear, reducing to $\C{x}[x\leftarrow \Cp{\abs{y}{y}}]=\C{x[x\leftarrow\abs{y}{y}}=\C{\abs{y}{y}}$.
\begin{schemeblock}
\begin{schemedisplay}
(lambda (flag)
  (lambda (marks)
    x))
\end{schemedisplay}
\end{schemeblock}
\end{schemedefinition}

\section{Definition of $\mathcal{C}$ in CPS}

We present a continuation mark transformation integrated in CPS.

\begin{schemedefinition}{\scheme|Ccps[(rator-expr rand-expr)]|}
\begin{schemeblock}
\begin{schemedisplay}
(lambda (kont)
   (lambda (flag)
     (lambda (marks)
       (((Ccps[rator-expr]
          (lambda (rator-value)
            (((Ccps[rand-expr]
               (lambda (rand-value)
                 ((((rator-value rand-value) kont) flag) marks)))
              FALSE) marks)))
         FALSE) marks))))
\end{schemedisplay}
\end{schemeblock}
\end{schemedefinition}

\begin{schemedefinition}{\scheme|C[(wcm mark-expr body-expr)]|}
\begin{schemeblock}
\begin{schemedisplay}
(lambda (kont)
  (lambda (flag)
    (lambda (marks)
      (((Ccps[mark-expr]
          (lambda (mark-value) 
            ((lambda (rest-marks) 
               (((Ccps[body-expr] kont) TRUE) Chcps[((CONS mark-value) rest-marks)]))
             ((flag Chcps[(SND marks)]) marks))))
        FALSE) marks))))
\end{schemedisplay}
\end{schemeblock}
\end{schemedefinition}

\begin{schemedefinition}{\scheme|Ccps[(ccm)]|}
\begin{schemeblock}
\begin{schemedisplay}
(lambda (kont)
  (lambda (flag)
    (lambda (marks)
      (kont marks))))
\end{schemedisplay}
\end{schemeblock}
\end{schemedefinition}

\begin{schemedefinition}{\scheme|Ccps[v]|=\scheme|Ccps[(lambda (x) e)]|}
\begin{schemeblock}
\begin{schemedisplay}
(lambda (kont)
  (lambda (flag)
    (lambda (marks)
      (kont (lambda (x) Ccps[e])))))
\end{schemedisplay}
\end{schemeblock}
\end{schemedefinition}

\begin{schemedefinition}{\scheme|Ccps[x]|}
\begin{schemeblock}
\begin{schemedisplay}
(lambda (kont)
  (lambda (flag)
    (lambda (marks)
      (kont x))))
\end{schemedisplay}
\end{schemeblock}
\end{schemedefinition}

We include corresponding definitions for $\mathcal{C}'$ and $\hat{\mathcal{C}}$.

\begin{schemedefinition}{\scheme|Cpcps[(lambda (x) e)]|}
\begin{schemeblock}
\begin{schemedisplay}
(lambda (x) Ccps[e])
\end{schemedisplay}
\end{schemeblock}
\end{schemedefinition}

\begin{schemedefinition}{\scheme|Chcps[p]|}
\begin{schemeblock}
\begin{schemedisplay}
(((Ccps[p] (lambda (x) x)) FALSE) Cpcps[nil])
\end{schemedisplay}
\end{schemeblock}
\end{schemedefinition}

\section{Example}

To better illustrate what the transformation does, we step through the reduction of a program which exhibits its more interesting aspects. One \cm\ program suited to this purpose is \scheme|(wcm 0 ((lambda (x) (wcm x (ccm))) 1))|. It reduces according to \cm\ semantics as
\begin{schemeblock}
\begin{schemedisplay}
(wcm 0 ((lambda (x) (wcm x (ccm))) 1))
(wcm 0 (wcm 1 (ccm)))
(wcm 1 (ccm))
(wcm 1 (lambda (z) ((z 1) (lambda (x) (lambda (y) y)))))
(lambda (z) ((z 1) (lambda (x) (lambda (y) y))))
\end{schemedisplay}
\end{schemeblock}

Now consider the reduction of the same program transformed. We apply the transformation just-in-time as we reduce to prevent term size explosion and promote clarity and omit uninteresting reductions.
\begin{schemeblock}
\begin{schemedisplay}
Ch[(wcm 0 ((lambda (x) (wcm x (ccm))) 1))]
\end{schemedisplay}
\end{schemeblock}

By definition this is
\begin{schemeblock}
\begin{schemedisplay}
((C[(wcm 0 ((lambda (x) (wcm x (ccm))) 1))] FALSE) Cp[NIL])
\end{schemedisplay}
\end{schemeblock}
\noindent
which explodes upon expansion to
\begin{schemeblock}
\begin{schemedisplay}
(((lambda (flag)
    (lambda (marks)
      ((C[((lambda (x) (wcm x (ccm))) 1)] TRUE)
       (((lambda (mark-value) (lambda (rest-marks) Cp[((CONS mark-value) rest-marks)]))
         ((C[0] FALSE) marks)) ((flag Ch[(SND marks)]) marks)))))
  FALSE) Cp[NIL])
\end{schemedisplay}
\end{schemeblock}

After the application of contextual information, we reach
\begin{schemeblock}
\begin{schemedisplay}
((C[((lambda (x) (wcm x (ccm))) 1)] TRUE)
 (((lambda (mark-value) (lambda (rest-marks) Cp[((CONS mark-value) rest-marks)]))
   ((C[0] FALSE) Cp[NIL])) ((FALSE Ch[(SND NIL)]) Cp[NIL])))
\end{schemedisplay}
\end{schemeblock}
\noindent
the transformation of the \scheme'wcm' body. Terms within are arranged so that correct evaluation occurs within the native call-by-value regime. This evaluates \scheme'mark-expr' and and prepends its value to the list of continuation marks before proceeding with evaluation of \scheme'body-expr'. This reduction soon yields the following term:
\begin{schemeblock}
\begin{schemedisplay}
((C[((lambda (x) (wcm x (ccm))) 1)] TRUE) Cp[((CONS 0) NIL)])
\end{schemedisplay}
\end{schemeblock}
\noindent
It is evident that this term will behave exactly as a top-level term except as this contextual information influences it, and this is exactly the property we have strived for. Expansion of this term yields
\begin{schemeblock}
\begin{schemedisplay}
(((lambda (flag)
    (lambda (marks)
      (((((C[(lambda (x) (wcm x (ccm)))] FALSE) marks)
         ((C[1] FALSE) marks))
        flag)
       marks))) TRUE) Cp[((CONS 0) NIL)])
\end{schemedisplay}
\end{schemeblock}
\noindent
the expansion of an application. In this example, both the operator and operand are values, so are essentially unaffected by the application of contextual information; this application has the effect of preparing the terms for application:
\begin{schemeblock}
\begin{schemedisplay}
((((lambda (x) C[(wcm x (ccm))])
   1) TRUE) Cp[((CONS 0) NIL)])
\end{schemedisplay}
\end{schemeblock}
\noindent
reduces to
\begin{schemeblock}
\begin{schemedisplay}
((C[(wcm 1 (ccm))]
   TRUE) Cp[((CONS 0) NIL)])
\end{schemedisplay}
\end{schemeblock}

This expands and reduces as the \scheme'wcm' term seen previously:
\begin{schemeblock}
\begin{schemedisplay}
(((lambda (flag)
    (lambda (marks)
      ((C[(ccm)] TRUE)
       (((lambda (mark-value) (lambda (rest-marks) Cp[((CONS mark-value) rest-marks)]))
         ((C[1] FALSE) marks))
        ((flag Ch[(SND marks)]) marks)))))
  TRUE) Cp[((CONS 0) NIL)])
\end{schemedisplay}
\end{schemeblock}

Of interest in this process is the effective collapse of the previous \scheme'mark' context by virtue of the value of \scheme'flag'. When we reach
\begin{schemeblock}
\begin{schemedisplay}
((lambda (marks) marks)
 ((lambda (rest-marks) Cp[((CONS 1) rest-marks)])
  ((TRUE Ch[(SND ((CONS 0) NIL))]) Cp[((CONS 0) NIL)])))
\end{schemedisplay}
\end{schemeblock}
\noindent
the list is beheaded to simulate mark overwriting:
\begin{schemeblock}
\begin{schemedisplay}
((lambda (marks) marks)
 ((lambda (rest-marks) Cp[((CONS 1) rest-marks)])
  Ch[(SND ((CONS 0) NIL))]))
\end{schemedisplay}
\end{schemeblock}

Once given the contextual information, the evaluation of \scheme'ccm' is simple:
\begin{schemeblock}
\begin{schemedisplay}
((lambda (marks) marks)
 Cp[((CONS 1) NIL)])
\end{schemedisplay}
\end{schemeblock}
\noindent
reduces to
\begin{schemeblock}
\begin{schemedisplay}
Cp[((CONS 1) NIL)]
\end{schemedisplay}
\end{schemeblock}
\noindent
and we are left with just what we hoped for.



