\chapter{\lc}

The \lc\ \cite{barendregt1984lambda} is a Turing-complete system of logic extensively used as a formal system for
expressing computation. Terms in the \lc\ are defined inductively; they take the form of
variables $x$ drawn from an infinite set, abstractions $\abs{x}{M}$ where $M$ is itself a
\lc\ term, and applications $\app{M}{N}$ where $M$ and $N$ are \lc\ terms.

Variables in the \lc\ are either \emph{free} or \emph{bound}. A variable $x$ is free if it does not
reside in the scope of a \emph{binding instance} of $x$. Otherwise, $x$ is bound. Terms
with no free variables can be called closed terms, combinators, or programs. The fact that
variables are drawn from an infinite set means that, given an arbitrary \lc\ term, we can
always obtain a \emph{fresh variable}, a variable not present in the term at hand. This is
critical as we will see shortly.

In an abstraction of the form $\abs{x}{M}$, $x$ is a binding instance which binds all free
occurrences of $x$ in the body $M$. To a first approximation, abstractions are functions.
For instance, the identity function can be expressed as $\abs{x}{x}$ where $x$ is
\emph{any} variable. Thus, there are an infinite number of ways to express the identity
function: $\abs{x}{x}$, $\abs{y}{y}$, $\abs{z}{z}$, etc. A consequence of this is that terms which are not syntactically equivalent may be semantically equivalent. This fact naturally gives rise to the notion of $\alpha$-equivalence which captures the idea that consistent renaming of bound variables, and their binding instances, does not change the meaning of a term.

There is some subtlety in valid renaming as there is the possibility of \emph{variable capture} which occurs when the new name of a binding instance is identical to that of some variable already present in the body. For instance, in the term $\abs{x}{y}$, if each $x$ is renamed to $y$, we obtain the term $\abs{y}{y}$, a fundamentally different term. For this reason, we need to take special care when we rename variables, which we will need to do regularly.

One of the ways abstractions approximate functions is that we can apply them to arguments.
This is signified simply by juxtaposition of function (or operator) and argument (or
operand). In the correct context, an application of the form $\app{M}{N}$ can be
\emph{reduced} in which the operator $M$ is applied to the operand $N$. For $M$ of the
form $\abs{x}{M'}$ for some $M'$, this entails the substitution of every free occurrence
of $x$ in $M'$ with $N$. The notation we adopt for this is $M'[x\leftarrow N]$. For
instance, the application $\app{\abs{x}{x}}{y}$ signifies $x[x\leftarrow y]$ and so
reduces to $y$. Here, we must protect against another strain of variable capture. As an
example, consider the reduction of $\app{\abs{x}{\abs{y}{x}}}{y}$. If we reduce naively,
we obtain $\abs{y}{y}$ which does not reflect the intended meaning of the reduction--the
argument $y$ has been captured by the abstraction, an act which destroys its meaning
within the environment. In order to avoid this, we must rename capturing abstractions in
$M$ to be outside the set of free variables of $N$. Within the greater term
$\app{\abs{x}{\abs{y}{x}}}{y}$, we rename $y$ to $z$ in $\abs{x}{\abs{y}{x}}$ obtaining
$\abs{x}{\abs{z}{x}}$. The subsequent reduction of $\app{\abs{x}{\abs{z}{x}}}{y}$ to
$\abs{z}{y}$ correctly reflects the intended meaning of the original term.

In the \lc, evaluation occurs during reduction, and reduction is merely application. There
is, however, yet more subtlety of which we must be aware: namely, in which contexts
applications are performed and terms evaluated. In the call-by-value \lc, denoted \lv,
evaluation of operands occurs before application. In contrast, in the call-by-name \lc,
denoted \la, application is performed as soon as the operator is resolved.

For instance, in $\app{\abs{x}{x}}{\app{\abs{y}{y}}{\abs{z}{z}}}$,
\begin{align*}
      &\app{\abs{x}{x}}{\app{\abs{y}{y}}{\abs{z}{z}}}\\
\lvrr &\app{\abs{x}{x}}{\abs{z}{z}}\\
\lvrr &\abs{z}{z}
\end{align*}
whereas
\begin{align*}
      &\app{\abs{x}{x}}{\app{\abs{y}{y}}{\abs{z}{z}}}\\
\larr &\app{\abs{y}{y}}{\abs{z}{z}}\\
\larr &\abs{z}{z}
\end{align*}
Although both terms reduce to the same term in this example, this distinction is not
merely pedantic: terms may reduce definitively in one reduction regime and fail to reduce
completely in the other! Consider
$\app{\abs{x}{\abs{y}{y}}}{\app{\abs{x}{\app{x}{x}}}{\abs{x}{\app{x}{x}}}}$ where
\[
\app{\abs{x}{\abs{y}{y}}}{\app{\abs{x}{\app{x}{x}}}{\abs{x}{\app{x}{x}}}}\larr\abs{y}{y}
\]
but
\begin{align*}
      &\app{\abs{x}{\abs{y}{y}}}{\app{\abs{x}{\app{x}{x}}}{\abs{x}{\app{x}{x}}}}\\
\lvrr &\app{\abs{x}{\abs{y}{y}}}{\app{\abs{x}{\app{x}{x}}}{\abs{x}{\app{x}{x}}}}\\
\lvrr &\cdots
\end{align*}
Historically at least, the call-by-name and call-by-value reduction regimes underlie the
distinction between so-called lazy and eager languages.

One final observation we should make about the \lc\ is which terms denote values. A value
should be, in a sense, irreducible and that criterion disqualifies applications from being
considered as values. A value should should not merely be a placeholder for arbitrary
values, and that criterion disqualifies lone variables from being considered as
values.\footnote{To be precise, a value is a \emph{closure}: an irreducible \lc\ term paired with an
environment which provides values for constituent free variables.} Thus, we shall consider
abstractions to be the sole form values can take in the \lc, habituating ourselves to the
idea that functions are data.

