\chapter{Continuation marks}

There are certain tools that are indispensable to some programmers that concern the
behavior of their programs: debuggers, profilers, steppers, etc. Without these tools,
these programmers cannot justify the adoption of a language, however compelling it might
otherwise be. Traditionally, these tools are developed at the same level the language is,
privy to incidental implementation detail, precisely because that detail enables these
tools to function. This is problematic for at least two reasons. First, it couples the
implementation of the tool with the implementation of the language, which increases the
cost to port to other platforms. If users become dependent upon these tools, it can stall
the advancement of the language and the adoption of new language features. Second and more
critical, it makes these tools unsound. For instance, debuggers typically examine programs
which have been compiled without optimizations. In general, this means that the debugged
program has different behavior than the deployed program. This is obviously undesirable.

It is desirable to implement such tools at the same level as the language, removing
dependency upon the implementation and instead relying on definitional and behavioral
invariants. Continuation marks are a language-level feature that provide the information
necessary for these tools to function. Furthermore, languages which require stack
inspection to enforce security policies (\emph{Java}, \emph{C\#}) or support aspect
oriented programming (\emph{aspectj}) can be defined in terms of a simpler language with
continuation marks \cite{clements2004tail, tucker2003pointcuts}.

Continuation marks originated in PLT Scheme (now Racket \cite{plt-tr1}) as a stack 
inspection mechanism. In fact, the \emph{Java} and \emph{C\#} languages rely on a similar 
stack inspection to enforce security policies of which continuation marks can be seen as 
a generalization. Surprisingly, continuation marks can be encoded in any language with 
exception facilities \cite{pettyjohn2005continuations}.

The feature of continuation marks itself is accessible via two surface level syntactic
forms: \scheme{with-continuation-mark} and \scheme{current-continuation-marks}.

In Scheme-like syntax, a \scheme{with-continuation-mark} expression appears as
\scheme|(with-continuation-mark key-expr value-expr body-expr)|. The evaluation of this
expression proceeds by first evaluating \scheme{key-expr} to \scheme{key} and
\scheme{value-expr} to \scheme{value}. Thereafter, \scheme{body-expr} is evaluated within
the extent of which \scheme'value' is associated with \scheme 'key'.

For instance, in the expression \scheme|(with-continuation-mark 'fun 'cons (cons 0
empty))|, the value \scheme{'+} is associated with the key \scheme{'fun} during the
evaluation of \scheme|(+ 2 2)|.

In the same Scheme-like syntax, a \scheme{current-continuation-marks} expression appears
as \scheme|(current-continuation-marks key-list)|. This expression evaluates to a list of
all the present key-value associations referenced in \scheme{key-list}. The
\scheme{with-continuation-mark} form admits a notion of ordering--inner associations are
more recent than outer ones--and this in reflected in the list yielded by the
\scheme{current-continuation-marks}.

Importantly, the result of \scheme{current-continuation-marks} provides no evidence of any
portion of the dynamic context lacking continuation marks with the specified keys. This
preserves the ability to perform optimizations without exposing details which would render
the optimizations unsound. This also requires special consideration of a language that
supports tail call optimization (which is not an optimization in the above sense since its
behavior is defined in the semantics of the language).

The canonical example to illustrate the behavior of continuation marks in the presence and
absence of tail call optimization is the factorial function.

Figure \ref{fig:fac-rec} illustrates the definitional recursive variant of the factorial
function. In this actualization, a cascade of multiplication operations builds as the
recursive calls are made. Each multiplication is pending computation of which the machine 
must keep track.

\begin{figure}
\begin{schemeblock}
\begin{schemedisplay}
(define (fac n)
  (if (= n 0)
      1
      (* n (fac (- n 1)))))
\end{schemedisplay}
\end{schemeblock}
\caption{The definitionally recursive factorial function}
\label{fig:fac-rec}
\end{figure}

Figure \ref{fig:fac-tail-rec} illustrates the tail recursive manifestation of the factorial
function. In contrast to the function in figure \ref{fig:fac-rec}, this formulation performs
the multiplication before the recursive call. Because the function has no pending
computations after the evaluation of the recursive call, the execution context need not
grow. Such a call is said to be in tail position.

\begin{figure}
\begin{schemeblock}
\begin{schemedisplay}
(define (fac-tr n acc)
  (if (= n 0)
      acc
      (fac-tr (- n 1) (* n acc))))
\end{schemedisplay}
\end{schemeblock}
\caption{A tail-recursive variant of the factorial function}
\label{fig:fac-tail-rec}
\end{figure}

Figures \ref{fig:fac-rec-cm} and \ref{fig:fac-tail-rec-cm} represent these two variants of the
factorial function augmented with continuation marks. Using these definitions, the 
result of \scheme|(fac 3)| would be

\begin{schemeblock}
\begin{schemedisplay}
(((fac 1)) ((fac 2)) ((fac 3)))
6
\end{schemedisplay}
\end{schemeblock}

whereas the result of \scheme|(fac-tr 3 1)| would be

\begin{schemeblock}
\begin{schemedisplay}
(((fac 1)))
6
\end{schemedisplay}
\end{schemeblock}

\begin{figure}
\begin{schemeblock}
\begin{schemedisplay}
(define (fac n)
  (if (= n 0)
      (begin
        (display (current-continuation-marks '(fac)))
        1)
      (with-continuation-mark 'fac n (* n (fac (- n 1)))))
\end{schemedisplay}
\end{schemeblock}
\caption{The definitionally recursive factorial function augmented with continuation marks}
\label{fig:fac-rec-cm}
\end{figure}

\begin{figure}
\begin{schemeblock}
\begin{schemedisplay}
(define (fac-tr n acc)
  (if (= n 0)
      (begin
        (display (current-continuation-marks '(fac)))
        acc)
      (with-continuation-mark 'fac n (fac-tr (- n 1) (* n acc))))
\end{schemedisplay}
\end{schemeblock}
\caption{The tail-recursive factorial function augmented with continuation marks}
\label{fig:fac-tail-rec-cm}
\end{figure}

This difference is due to the growing continuation in the definitionally recursive
\scheme{fac}. Each call to \scheme{fac} has a pending computation--namely, the
multiplication--after the recursive call and so each necessitates the creation of
additional evaluation context. The effect of this additional context is that each
annotation is applied to new, ``blank'' context, so all the previous annotations are
preserved. In the tail-recursive variant, there is no pending computation and therefore no
additional evaluation context. In this instance, the previous mark is overwritten with the
new.

