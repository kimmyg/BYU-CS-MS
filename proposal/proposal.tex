\documentclass[ms]{byuprop}

% Options for this class include the following (* indicates default): 
%   10pt -- 10 point font size 
%   11pt -- 11 point font size 
%   12pt (*) -- 12 point font size 
%
%   ms -- produce a thesis proposal (off) 
%   areaexam -- produce a research area overview (off) 
%   phd -- produce a dissertation proposal (off) 
%
%   layout -- show layout lines on the pages, helps with overfull boxes (off) 
%   grid -- show a half-inch grid on every page, helps with printing (off)

% This command fixes my particular printer, which starts 0.03 inches too low, 
% shifting the whole page down by that amount. This shifts the document 
% content up so that it comes out right when printed. 
%
% Discovering this sort of behavior is best done by specifying the ``grid'' 
% option in the class parameters above.  It prints a 1/2 inch grid on every 
% page. You can then use a ruler to determine exactly what the printer is 
% doing. 
% Uncomment to shift content up (accounting for printer problems)
%\setlength{\voffset}{-.03in}

% Here we set things up for invisible hyperlinks in the document. This makes 
% the electronic version clickable without changing the way that the document 
% prints. It's useful, but optional. Note that if you use pdflatex, you 
% should change "ps2pdf" to "pdftex". 
\usepackage[
	pdftex,
	bookmarks=true,
	breaklinks=true,
	raiselinks=true,
	pdfborder={0 0 0},
	colorlinks=false,
	]{hyperref}

% Rewrite the itemize, description, and enumerate environments to have more 
% reasonable spacing: 
\newcommand{\ItemSep}{\itemsep 0pt}
\let\oldenum=\enumerate
\renewcommand{\enumerate}{\oldenum \ItemSep}
\let\olditem=\itemize
\renewcommand{\itemize}{\olditem \ItemSep}
\let\olddesc=\description
\renewcommand{\description}{\olddesc \ItemSep}

% Get a little less fussy about word spacing on a line.  Sometimes produces 
% ugly results, so keep your eyes peeled. 
\sloppy

% Important settings for the byuprop class. %
%%%%%%%%%%%%%%%%%%%%%%%%%%%%%%%%%%%%%%%%%%%%%

% Because I use these things in more than one place, I created new commands for 
% them. I did not use \providecommand because I absolutely want LaTeX to error 
% out if these already exist. 

\newcommand{\Title}{A Correct Compiler for Continuation Marks using a CPS Transformation}
\newcommand{\Author}{Kimball R. Germane} 
\newcommand{\SubmissionMonth}{May}
\newcommand{\SubmissionYear}{2012}

% Take these from the commands defined above
\title{\Title}
\author{\Author}
\monthsubmitted{\SubmissionMonth} 
\yearsubmitted{\SubmissionYear}

% Committee members 
\committeechair{Jay McCarthy} 
\committeemembera{Sean Warnick} 
\committeememberb{}
\committeememberc{} 
\committeememberd{}

% Department graduate coordinator 
\graduatecoordinator{Dan Ventura}

%%%%%%%%%%%%%%%%%%%%%%%%%%%%%%%%%%%%%%%%%%%%%

% Set up the internal PDF information so that it becomes part of the document 
% metadata. The pdfinfo command will display this. Be sure to set the document 
% type and add your own keywords. 
\hypersetup{%
	pdftitle=\Title,%
	pdfauthor=\Author,%
	pdfsubject={Document Type, BYU CS Department: %
				Submitted \SubmissionMonth~\SubmissionYear, Created \today},%
	pdfkeywords={},%
}

% These packages allow the bibliography to be sorted alphabetically and allow references
% to more than one paper to be sorted and compressed (i.e. instead of [5,2,4,6] you get
% [2,4-6]) 

%\usepackage[numbers,sort&compress]{natbib}
%\usepackage{hypernat}

% Additional packages required for your specific thesis go here. I've left some I use as
% examples. 
%\usepackage{graphicx}
%\usepackage{pdfsync}
\usepackage{amsmath}
\usepackage{amsthm}
\usepackage{float}

\floatstyle{boxed} 
\restylefloat{figure}

\newcounter{definition}
\newcounter{example}

\begin{document}

% Produce the preamble
\maketitle

%essence of compiling with continuations, Flanagan et al.
%compiling with continuations, Appel
%Call by name, call by value, and the lambda calculus, Plotkin
%Compiling with continuations, continued, Kennedy
%PLAI

%\abstract{}

\section{Introduction}

% remember that the audience in this case actually does not know what these things are 
% so have a clear, useful exposition.

Thesis: A CPS-style global transformation can compile the $\lambda$-calculus with
continuation marks into the $\lambda$-calculus in a semantics-preserving
way.

% talk more about how they are useful here, elaboration will be below



Continuation marks \cite{clements2006portable} are a language feature that provides a
mechanism to annotate the dynamic context of a program. This feature allows the
association of arbitrary values with arbitrary keys for the lifetime of an execution
context and the inquiry of currently defined values of pending contexts for a given set of
keys. This is advantageous for programs that require dynamic information about a program
execution such as debuggers, profilers, and steppers because it allows them to be defined
at the same level as the language instead of some level below.

Continuation marks originated in PLT Scheme (now Racket) \cite{plt-tr1} as a stack 
inspection mechanism. In fact, the Java and C\# languages rely on a similar stack 
inspection to enforce security policies, and continuation marks can be seen as a 
generalization of that. Surprisingly, continuation marks are supported by any language 
with exception facilities \cite{pettyjohn2005continuations} which fact has led to their 
experimental addition to Javascript \cite{clements2008implementing}.

The continuation-passing style (CPS) transformation is actually a family of 
transformations designed to make certain analyses simpler. The vanilla CPS transformation 
augments each function with an additional formal parameter, the \emph{continuation}, which 
is a functional representation of currently pending computation. Functions in CPS never 
return; instead, they call the continuation argument with their result. The CPS 
transformation then simplifies programs by representing all control and data transfer 
uniformly and explicitly.

The ``spirit'' of the CPS transformation is to represent all transfers of control 
uniformly \cite{sabry1994formal}. When a given transformation has that property, one can 
say that it is in the family of CPS transformations. For instance, the ``double-barrelled'' 
CPS transformation is a variant of the standard wherein each function receives not one but 
two additional formal parameters, each a continuation. One application of this particular 
transformation is error handling, one continuation argument representing the remainder of 
a successful computation, the other representing the failure contingency. It is especially 
useful in modelling exceptions or other non-local transfers of control in situations where 
the computation might fail.

%It is our interest to understand the essence of continuation marks--their behavior in the 
%absense of other language features and implementation details. For this, we take the core 
%of computation, the $\lambda$-calculus, and add facilities to manipulate continuation 
%marks. These two together comprise a language which we term $\lambda_{cm}$. Our approach 
%of determining what continuation marks mean computationally will be to express $\lambda_{cm}$ 
%in the $\lambda$-calculus which preserves the meaning of the language. Our method to make 
%such an expression will be to perform a transformation in the spirit of CPS which will make 
%certain useful analyses accessible.

%Because a continuation call is the sole method of control and its argument is the sole methwhich make explicit all data and control transfer

%The language semantics is defined but no correctness proof of the interpreter exists so there is no way to determine whether it is behaving correctly.

%There is a history of transforming languages 

%Continuation-passing style is a breed of languages targeted by transformational compilers 
%amenable to analysis, optimization, verification. (?) 

%Programs in CPS are simpler in that all transfer of control is uniformly represented, which simplifies the construction of compilers \cite{sabry1994formal}.

%CPS has a close correspondence to an imperative instruction sequence which eases translation to low-level machines \cite{appel2007compiling}.

%CPS is more amenable to reasoning and proof [by virtue of what? it's uniformity?] which...
%one can get more semantics \cite{strachey2000continuations}
%more optimizations are possible \cite{plotkin1975call}



%is a characterization of code in which an additional formal 
%parameter, the \emph{continuation}, is added to each function definition. This parameter 
%is itself a function which accepts one argument. Instead of returning, functions call 
%the continuation with their result as the argument. In computation represented in 
%CPS, consequently, all control- and data-flow is explicit.

\section{Related Work}

\subsection{Continuation marks}

There are certain tools that are indispensable to some programmers that concern the
behavior of their programs: debuggers, profilers, steppers, etc. Without these tools,
these programmers cannot justify the adoption of a language, however compelling it might
otherwise be. Traditionally, these tools are developed at the same level as the 
language, privy to incidental implementation detail, precisely because that detail 
enables these tools to function. This is problematic for at least two reasons. First, 
it couples the implementation of the tool with the implementation of the language, which
increases the cost to port to other platforms. If users become dependent upon these tools,
it can stall the advancement of the language and the adoption of new language features.
Second and more critical, it makes these tools unsound. For instance, debuggers typically
examine programs which have been compiled without optimizations. In general, this means 
that the debugged program has different behavior than the deployed program. This is 
obviously undesirable.

It is desirable to implement such tools at the same level as the language, removing
dependency upon the implementation an instead relying on definitional and behavioral
invariants. Continuation marks are a language-level feature that provide the information
necessary for these tools to function. Furthermore, languages which require stack
inspection to enforce security policies (\emph{Java}, \emph{C\#}) or support aspect
oriented programming (\emph{aspectj}) can be defined in terms of a simpler language with
continuation marks.

The feature of continuation marks itself is accessible via two surface level syntactic
forms: \emph{with-continuation-mark} and \emph{current-continuation-marks}.

\emph{with-continuation-mark} has three parameters: a key expression \emph{key-expr}, a 
value expression \emph{value-expr} and a body expression \emph{body-expr}. The evaluation 
of \emph{value-expr} will be associated with a key, the evaluation of \emph{key-expr}, 
before the evaluation of \emph{body-expr}. During the lifetime of the evaluation of 
\emph{body-expr}, a continuation mark will exist associated with this key. In a Scheme-like 
syntax, this call appears like so:

% will use slatex for the dissertation; too much of a hassle now
\texttt{(with-continuation-mark} \emph{key-expr} \emph{value-expr} \emph{body-expr}\texttt{)}

\emph{current-continuation-marks} has one parameter, a set of keys \emph{key-set}, and returns a list of
all the associated values attached to the dynamic context of the invocation. If a particular 
context has been annotated by more than one key in the set, this will be reflected in the 
returned list\footnote{The returned list is typically a list of non-empty lists where each 
sub-list represents the marks on a context.}. Additionally, the order in which values were 
attached with a particular key is preserved. Scheme-like, \emph{current-continuation-marks}
looks like this:

\texttt{(current-continuation-marks} \emph{key-set}\texttt{)}

Importantly, the result of \emph{current-continuation-marks} provides no evidence of any portion of the
dynamic context lacking continuation marks with the specified keys. This preserves the
ability to perform optimizations without exposing details which would render the
optimizations unsound. This also requires special consideration of a language that
supports tail call optimization (which is not an optimization in the above sense since its
behavior is defined in the semantics of the language). By definition, \emph{body-expr} is 
in tail position; a language with tail call optimization will reflect this.

The canonical example to illustrate the behavior of continuation marks in the presence and
absence of proper tail recursion is the factorial function.

Figure \ref{fac-rec} illustrates the definitional recursive variant of the factorial
function. In this actualization, a cascade of multiplication operations builds as the
recursive calls are made. Each multiplication is computation that must be performed after
the recursive call of which the machine must keep track.

\begin{figure}
\begin{verbatim}
(define (fact n)
  (if (= n 0)
      1
      (* n (fact (- n 1)))))
\end{verbatim}
\caption{The definitionally recursive factorial function}
\label{fac-rec}
\end{figure}

Figure \ref{fac-tail-rec} illustrates the tail recursive manifestation of the factorial
function. In contrast to the function in figure \ref{fac-rec}, this formulation performs
the multiplication before the recursive call. Because the function has no pending
computations after the evaluation of the recursive call, the execution context need not
grow. Such a call is said to be in tail position.

\begin{figure}
\begin{verbatim}
(define (fact-tr n acc)
  (if (= n 0)
      acc
      (fact-tr (- n 1) (* n acc))))
\end{verbatim}
\caption{A tail-recursive variant of the factorial function}
\label{fac-tail-rec}
\end{figure}

Figures \ref{fac-rec-cm} and \ref{fac-tail-rec-cm} represent these two variants of the
factorial function augmented with continuation marks.


\begin{figure}
\begin{verbatim}
(define (fact n)
  (if (= n 0)
      (begin
        (display (current-continuation-marks '(fact)))
        1)
      (with-continuation-mark 'fact n (* n (fact (- n 1)))))
\end{verbatim}
\caption{The definitionally recursive factorial function augmented with continuation marks}
\label{fac-rec-cm}
\end{figure}

\begin{figure}
\begin{verbatim}
(define (fact-tr n acc)
  (if (= n 0)
      (begin
        (display (current-continuation-marks '(fact)))
        acc)
      (with-continuation-mark 'fact n (fact-tr (- n 1) (* n acc))))
\end{verbatim}
\caption{The tail-recursive factorial function augmented with continuation marks}
\label{fac-tail-rec-cm}
\end{figure}

The result of \texttt{(fact 3)} would be

\begin{verbatim}
(((fact 1)) ((fact 2)) ((fact 3)))
6
\end{verbatim}

whereas the result of \texttt{(fact-tr 3 1)} would be

\begin{verbatim}
(((fact 1)))
6
\end{verbatim}

This difference is due to the growing continuation in the definitionally recursive
\texttt{fact}. Each call to \texttt{fact} has a pending computation--namely, the
multiplication--after the recursive call and so each necessitates the creation an
additional evaluation context. The effect of these additional contexts is that each
annotation is applied to a new, ``blank'' context, so all the annotations are preserved. In
the tail-recursive variant, there is no pending computation and therefore no additional
evaluation context. In this instance, the previous mark is overwritten with the new.

\subsection{Continuation-passing style}

%what it is

Because all control- and data-flow is made explicit, CPS is especially useful as an 
intermediate representation \cite{flanagan1993essence}. Take a simple computation such as

Some programming environments require an inversion of control which is cumbersome for the programmer. This should make it immediately apparent that this is a language problem. The web is an illustrative example of this. A user's interaction with a web application may involve a series of self-contained submissions to a server. The web environment makes available the possibility of regression or duplication of interaction (via the browser's ``Back'' button, for example) so correct server applications must be designed to deal with multiple and concurrent interactions \cite{queinnec2003inverting}.
%useful in reinverting control (PLAI)
%talk about double-barreled cps and exceptions

One of the variants of CPS augments each function definition with \emph{two} formal 
parameters and each call site with two arguments: the success continuation and the 
failure continuation. The success continuation is treated as the primary one, meaning 
that it embodies the remainder of work performed in a successful computation. The 
failure continuation is passed into subcalls as is or wrapped appropriately.

Consider the following definition of a division function
\begin{verbatim}
(define (div a b k e)
  (if (= b 0)
      (e ?)
      (k (/ a b))))
\end{verbatim}
wherein \texttt{/} denotes a primitive division function which will fail when the 
denominator is zero.

% find better example for failure continuations

\begin{verbatim}
(/ 6 3 k e)
\end{verbatim}

From this, we can see that failure continuations behave like exceptions and indeed 
represent them in continuation-passing style.

This and similar variants of CPS in which two continuations are represented are 
referred to colloquially as ``double-barrelled'' CPS.

CPS is the ideal form to understand tail-call optimization. In it, the fact that the 
continuation doesn't grow (as opposed to the stack which is an upper bound for the 
continuation) is visible/explicit. 

Consider once again the two variants of the factorial function, sans continuation marks, 
given earlier. In CPS, the standard definitional variant can be expressed as

\begin{verbatim}
fact n k = 
  if n == 0
    k 1
  else
    fact (n - 1) (lambda (acc) (k (* n acc)))
\end{verbatim}

and the tail-recursive variant as

\begin{verbatim}
fact_tr n acc k =
  if n == 0
    k acc
  else
    fact (n - 1) (n * acc) k
\end{verbatim}.

Notice that, in the first variation, each recursive call receives a newly-constructed 
$k$ encapsulating additional work to be performed at the completion of the recursive 
computation. In the second, $k$ is passed unmodified, so while computation occurs 
within each context, no \emph{additional} computation pends.

---

CPS transformations do not typically distinguish between ``primitive'' functions and ``non-primitive'' functions meaning that operations are decomposed by the transformation until raw values are obtained. Consequently, seemingly simple functions--so simple that they are seldom treated as functions--are victim [this sentiment, different word, weaker] to the transform.
%(This may be the method by which the call-by-value is simulated by the call-by-name--or somehow related to it. If this is the case, it is probably found in Plotkin's paper. If so, bring it up.)

\section{Project description}

The source language is the $\lambda$-calculus with two additional forms specific to continuation 
marks. Terms in this language are either single variables; $\lambda$-abstractions; function 
applications (all these being inherited from the $\lambda$-calculus); the \emph{wcm} form, an 
abbreviation of \emph{with-continuation-mark}; and the \emph{ccm} form, an abbreviation of 
\emph{current-continuation-marks}. We will call this language $\lambda_{cm}$.

go from compiling language with feature to language without feature -> 

there is some paper by wadler and ?
continuation marks as exceptions
delimited continuations in CPS

things to include:
history of CPS
compilers of continuation marks
cite Racket
talk about continuation marks in Javascript


talk about call/cc as the quintessential use of CPS

\begin{align*}
M ::= &x\\
      &\lambda x.M\\
      &M\,M\\
      &\mathrm{wcm}\,M\,M\\
      &\mathrm{ccm}\\
\end{align*}

We will provide a CPS transformation of the $\lambda$-calculus with continuation marks.

\subsection{Language Definition}

Figure \ref{language-definition} presents the syntactic forms of the language, the 
definition of a metafunction $\chi$ in terms of those forms, and the definitions of 
expressions and values in the language. Definitions of $E$ and $F$ signify contexts in the 
computation. The distinction between $E$ and $F$ is made to force directly nested 
\emph{wcm} terms to behave as solely the innermost term. The definition of $e$ establishes 
the forms of valid expressions in the language: applications, variables, values, \emph{wcm} 
forms, and \emph{ccm} forms. The definition of $v$ denotes that values in the language 
are $\lambda$-abstractions.

%definitional constraint: the evaluation of the mark occurs before the evaluation of the 
%body. (This constraint is crucial in the face of mutation.) The corresponding rules in the 
%right column describe the evaluation of the $\chi$ metafunction upon which the evaluation 
%of the \emph{ccm} form depends.

\begin{figure}
\begin{align*}
E = &(\mathrm{wcm}\,v\,F) & \chi((\mathrm{wcm}\,v\,F)) &= v : (\chi(F))\\
    &F\\
F = &[]                   & \chi([])                   &= \mathrm{empty}\\
    &(E\,e)               & \chi((E\,e))               &= \chi(E)\\
    &(v\,E)               & \chi((v\,E))               &= \chi(E)\\
    &(\mathrm{wcm}\,E\,e) & \chi((\mathrm{wcm}\,E\,e)) &= \chi(E)\\
e = &(e\,e)\\
    &x\\
    &v\\
    &(\mathrm{wcm}\,e\,e)\\
    &(\mathrm{ccm})\\
v = & \lambda x. e
\end{align*}
\caption{Grammar and the $\chi$ metafunction}
\label{language-definition}
\end{figure}

The definitions in figure \ref{language-semantics} establish the proper interpretation of 
various expressions. The first follows the typical definition of application. The second 
defines the tail behavior of the \emph{wcm} form. The third expresses that the \emph{wcm} 
form takes on the value of its body. Finally, the fourth defines the value of the \emph{ccm} 
form in terms of the $\chi$ metafunction.

\begin{figure}
\begin{align*}
E[(\lambda x.e)\,v]                         &\rightarrow E[e[x\leftarrow v]]\\
E[(\mathrm{wcm}\,v\,(\mathrm{wcm}\,v'\,e))] &\rightarrow E[(\mathrm{wcm}\,v'\,e)]\\
E[(\mathrm{wcm}\,v\,v')]                    &\rightarrow E[v']\\
E[(\mathrm{ccm})]                           &\rightarrow E[\chi(E)]
\end{align*}
\caption{Evaluation rules}
\label{language-semantics}
\end{figure}

\section{Validation}

If an expression reduces to a value according to the language semantics, the expression ``compiled'' to the $\lambda$-calculus reduces to the value ``compiled'' to the $\lambda$-calculus according to the $\lambda$-calculus semantics.

\[
e\rightarrow^{*}_{cm}v\Rightarrow c(e)\rightarrow^{*}_{\lambda}c(v)
\]

\section{Thesis schedule}

Less than five years.

%Abstract – 1 to 2 paragraphs summarizing the proposal. Introduction – 1 to 4 pages
%answering questions 1 and 2 above Related Work – 1 to 2 pages answering question 3 above.
%Thesis statement – 1 to 2 sentences stating what is to be demonstrated in your thesis.
%Project Description – 2 to 5 pages answering question 4 above. Validation – 1/2 to 2 pages
%answering question 5 above. Thesis Schedule – ¼ to ½ page specifying dates for completion
%of major milestones. Annotated Bibliography – 2 to 5 pages containing references for all
%work cited.

%%%%%%%%%%%%%%%%%%%%%%%%%

% Change these to reflect the bibliography style and bibtex database file you want to use
\bibliographystyle{annotate}
\bibliography{proposal}

\end{document}
