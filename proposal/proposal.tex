\documentclass[ms]{byuprop}

% Options for this class include the following (* indicates default): 
%   10pt -- 10 point font size 
%   11pt -- 11 point font size 
%   12pt (*) -- 12 point font size 
%
%   ms -- produce a thesis proposal (off) 
%   areaexam -- produce a research area overview (off) 
%   phd -- produce a dissertation proposal (off) 
%
%   layout -- show layout lines on the pages, helps with overfull boxes (off) 
%   grid -- show a half-inch grid on every page, helps with printing (off)

% This command fixes my particular printer, which starts 0.03 inches too low, 
% shifting the whole page down by that amount. This shifts the document 
% content up so that it comes out right when printed. 
%
% Discovering this sort of behavior is best done by specifying the ``grid'' 
% option in the class parameters above.  It prints a 1/2 inch grid on every 
% page. You can then use a ruler to determine exactly what the printer is 
% doing. 
% Uncomment to shift content up (accounting for printer problems)
%\setlength{\voffset}{-.03in}

% Here we set things up for invisible hyperlinks in the document. This makes 
% the electronic version clickable without changing the way that the document 
% prints. It's useful, but optional. Note that if you use pdflatex, you 
% should change "ps2pdf" to "pdftex". 
\usepackage[
	pdftex,
	bookmarks=true,
	breaklinks=true,
	raiselinks=true,
	pdfborder={0 0 0},
	colorlinks=false,
	]{hyperref}

% Rewrite the itemize, description, and enumerate environments to have more 
% reasonable spacing: 
\newcommand{\ItemSep}{\itemsep 0pt}
\let\oldenum=\enumerate
\renewcommand{\enumerate}{\oldenum \ItemSep}
\let\olditem=\itemize
\renewcommand{\itemize}{\olditem \ItemSep}
\let\olddesc=\description
\renewcommand{\description}{\olddesc \ItemSep}

% Get a little less fussy about word spacing on a line.  Sometimes produces 
% ugly results, so keep your eyes peeled. 
\sloppy

% Important settings for the byuprop class. %
%%%%%%%%%%%%%%%%%%%%%%%%%%%%%%%%%%%%%%%%%%%%%

% Because I use these things in more than one place, I created new commands for 
% them. I did not use \providecommand because I absolutely want LaTeX to error 
% out if these already exist. 

\newcommand{\Title}{A Semantics-Preserving CPS-like Transformation for Continuation Marks}
\newcommand{\Author}{Kimball R. Germane} 
\newcommand{\SubmissionMonth}{May}
\newcommand{\SubmissionYear}{2012}

% Take these from the commands defined above
\title{\Title}
\author{\Author}
\monthsubmitted{\SubmissionMonth} 
\yearsubmitted{\SubmissionYear}

% Committee members 
\committeechair{Jay McCarthy} 
\committeemembera{Sean Warnick} 
\committeememberb{Dennis Ng}
%\committeememberc{} 
%\committeememberd{}

% Department graduate coordinator 
\graduatecoordinator{Dan Ventura}

%%%%%%%%%%%%%%%%%%%%%%%%%%%%%%%%%%%%%%%%%%%%%

% Set up the internal PDF information so that it becomes part of the document 
% metadata. The pdfinfo command will display this. Be sure to set the document 
% type and add your own keywords. 
\hypersetup{%
	pdftitle=\Title,%
	pdfauthor=\Author,%
	pdfsubject={Document Type, BYU CS Department: %
				Submitted \SubmissionMonth~\SubmissionYear, Created \today},%
	pdfkeywords={},%
}

% These packages allow the bibliography to be sorted alphabetically and allow references
% to more than one paper to be sorted and compressed (i.e. instead of [5,2,4,6] you get
% [2,4-6]) 

%\usepackage[numbers,sort&compress]{natbib}
%\usepackage{hypernat}

% Additional packages required for your specific thesis go here. I've left some I use as
% examples. 
%\usepackage{graphicx}
%\usepackage{pdfsync}
\usepackage{amsmath}
\usepackage{amsthm}
\usepackage{float}

\floatstyle{boxed} 
\restylefloat{figure}

\newcounter{definition}
\newcounter{example}

\newtheorem*{cmtheorem}{$\mathcal{C}:\lambda_{CM}\rightarrow\lambda$ Meaning Preservation Theorem}

\begin{document}

% Produce the preamble
\maketitle

%essence of compiling with continuations, Flanagan et al.
%compiling with continuations, Appel
%Call by name, call by value, and the lambda calculus, Plotkin
%Compiling with continuations, continued, Kennedy
%PLAI

%I assume a thesis proposal should be written so that a reader who is not familiar with 
%your area of study is given a good chance to understand the (i) problem that you propose 
%to solve, (ii) significance of the problem, (iii) proposed solution of the problem, and 
%(iv) arguments that the solution should solve the problem. Of course the proposal 
%should also justify the merit and uniqueness of your proposed solution. 
%If any of these discussions is missing, then the proposal is incomplete and should be 
%further revised accordingly.

\abstract{Continuation marks are a programming language feature which generalize stack
inspection. Continuation marks currently lack a meaning-preserving transformation to the
$\lambda$-calculus, a useful and widely-used computation model. Such a transformation
would simplify the construction of compilers which treat continuation marks correctly. We
will present a CPS-like transformation from the call-by-value $\lambda$-calculus augmented
with continuation marks to the pure call-by-value $\lambda$-calculus and verify that the
transformation indeed preserves meaning.}

\section{Introduction}

% remember that the audience in this case actually does not know what these things are 
% so have a clear, useful exposition.

Thesis: A CPS-like global transformation can compile the $\lambda$-calculus with
continuation marks into the plain $\lambda$-calculus in a semantics-preserving way.

Continuation marks \cite{clements2006portable} are a language feature that provides a
mechanism to annotate the dynamic context of a program. This feature allows the
association of arbitrary values with arbitrary keys for the lifetime of an execution
context and the inquiry of currently defined values of pending contexts for a given set of
keys. This is advantageous for programs that require dynamic information about a program
execution such as debuggers, profilers, and steppers because it allows them to be defined
at the same level as the language instead of some level below.

The continuation-passing style (CPS) transformation is actually a family of 
transformations designed to make certain analyses simpler. The simplest variation of the 
CPS transformation augments each function with an additional formal parameter, the 
\emph{continuation}, which is a functional representation of currently pending 
computation. Functions in CPS never return; instead, they call the continuation argument 
with their result. The CPS transformation then simplifies programs by representing all 
control and data transfer uniformly and explicitly. In general, the ``spirit'' of the 
CPS transformation is to represent all transfers of control uniformly \cite{sabry1994formal}.

It is our interest to understand the essence of continuation marks--their behavior in the 
absense of other language features and implementation details. For this, we take the core 
of computation, the $\lambda$-calculus, and add facilities to manipulate continuation 
marks. These two together comprise a language which we term $\lambda_{cm}$. By expressing 
$\lambda_{cm}$ in terms of the plain $\lambda$-calculus, we uncover the meaning of 
continuation marks in a pure computational language. We arrive at this expression by 
performing a transformation in the spirit of CPS from $\lambda_{cm}$ to the $\lambda$-calculus 
verified to preserve the meaning of the source language.

\section{Related Work}

\subsection{Continuation marks}

There are certain tools that are indispensable to some programmers that concern the
behavior of their programs: debuggers, profilers, steppers, etc. Without these tools,
these programmers cannot justify the adoption of a language, however compelling it might
otherwise be. Traditionally, these tools are developed at the same level as the 
language, privy to incidental implementation detail, precisely because that detail 
enables these tools to function. This is problematic for at least two reasons. First, 
it couples the implementation of the tool with the implementation of the language, which
increases the cost to port to other platforms. If users become dependent upon these tools,
it can stall the advancement of the language and the adoption of new language features.
Second and more critical, it makes these tools unsound. For instance, debuggers typically
examine programs which have been compiled without optimizations. In general, this means 
that the debugged program has different behavior than the deployed program. This is 
obviously undesirable.

It is desirable to implement such tools at the same level as the language, removing
dependency upon the implementation an instead relying on definitional and behavioral
invariants. Continuation marks are a language-level feature that provide the information
necessary for these tools to function. Furthermore, languages which require stack
inspection to enforce security policies (\emph{Java}, \emph{C\#}) or support aspect
oriented programming (\emph{aspectj}) can be defined in terms of a simpler language with
continuation marks.

Continuation marks originated in PLT Scheme (now Racket \cite{plt-tr1}) as a stack 
inspection mechanism. In fact, the \emph{Java} and \emph{C\#} languages rely on a similar 
stack inspection to enforce security policies of which continuation marks can be seen as 
a generalization. Surprisingly, continuation marks can be encoded in any language with 
exception facilities \cite{pettyjohn2005continuations} which fact has led to their 
experimental addition to Javascript \cite{clements2008implementing}.

The feature of continuation marks itself is accessible via two surface level syntactic
forms: \emph{with-continuation-mark} and \emph{current-continuation-marks}.

\emph{with-continuation-mark} has three parameters: a key expression \emph{key-expr}, a 
value expression \emph{value-expr} and a body expression \emph{body-expr}. The evaluation 
of \emph{value-expr} will be associated with a key, the evaluation of \emph{key-expr}, 
before the evaluation of \emph{body-expr}. During the lifetime of the evaluation of 
\emph{body-expr}, a continuation mark will exist associated with this key. In a Scheme-like 
syntax, this call appears like so:

% will use slatex for the dissertation; too much of a hassle now
\texttt{(with-continuation-mark} \emph{key-expr} \emph{value-expr} \emph{body-expr}\texttt{)}

\emph{current-continuation-marks} has one parameter, a set of keys \emph{key-set}, and returns a list of
all the associated values attached to the dynamic context of the invocation. If a particular 
context has been annotated by more than one key in the set, this will be reflected in the 
returned list\footnote{The returned list is typically a list of non-empty lists where each 
sub-list represents the marks on a context.}. Additionally, the order in which values were 
attached with a particular key is preserved. Scheme-like, \emph{current-continuation-marks}
looks like this:

\texttt{(current-continuation-marks} \emph{key-set}\texttt{)}

Importantly, the result of \emph{current-continuation-marks} provides no evidence of any portion of the
dynamic context lacking continuation marks with the specified keys. This preserves the
ability to perform optimizations without exposing details which would render the
optimizations unsound. This also requires special consideration of a language that
supports tail call optimization (which is not an optimization in the above sense since its
behavior is defined in the semantics of the language). By definition, \emph{body-expr} is 
in tail position; a language with tail call optimization will reflect this.

The canonical example to illustrate the behavior of continuation marks in the presence and
absence of proper tail recursion is the factorial function.

Figure \ref{fac-rec} illustrates the definitional recursive variant of the factorial
function. In this actualization, a cascade of multiplication operations builds as the
recursive calls are made. Each multiplication is computation that must be performed after
the recursive call of which the machine must keep track.

\begin{figure}
\begin{verbatim}
(define (fact n)
  (if (= n 0)
      1
      (* n (fact (- n 1)))))
\end{verbatim}
\caption{The definitionally recursive factorial function}
\label{fac-rec}
\end{figure}

Figure \ref{fac-tail-rec} illustrates the tail recursive manifestation of the factorial
function. In contrast to the function in figure \ref{fac-rec}, this formulation performs
the multiplication before the recursive call. Because the function has no pending
computations after the evaluation of the recursive call, the execution context need not
grow. Such a call is said to be in tail position.

\begin{figure}
\begin{verbatim}
(define (fact-tr n acc)
  (if (= n 0)
      acc
      (fact-tr (- n 1) (* n acc))))
\end{verbatim}
\caption{A tail-recursive variant of the factorial function}
\label{fac-tail-rec}
\end{figure}

Figures \ref{fac-rec-cm} and \ref{fac-tail-rec-cm} represent these two variants of the
factorial function augmented with continuation marks. Using these definitions, the 
result of \texttt{(fact 3)} would be

\begin{verbatim}
(((fact 1)) ((fact 2)) ((fact 3)))
6
\end{verbatim}

whereas the result of \texttt{(fact-tr 3 1)} would be

\begin{verbatim}
(((fact 1)))
6
\end{verbatim}.

\begin{figure}
\begin{verbatim}
(define (fact n)
  (if (= n 0)
      (begin
        (display (current-continuation-marks '(fact)))
        1)
      (with-continuation-mark 'fact n (* n (fact (- n 1)))))
\end{verbatim}
\caption{The definitionally recursive factorial function augmented with continuation marks}
\label{fac-rec-cm}
\end{figure}

\begin{figure}
\begin{verbatim}
(define (fact-tr n acc)
  (if (= n 0)
      (begin
        (display (current-continuation-marks '(fact)))
        acc)
      (with-continuation-mark 'fact n (fact-tr (- n 1) (* n acc))))
\end{verbatim}
\caption{The tail-recursive factorial function augmented with continuation marks}
\label{fac-tail-rec-cm}
\end{figure}

This difference is due to the growing continuation in the definitionally recursive
\texttt{fact}. Each call to \texttt{fact} has a pending computation--namely, the
multiplication--after the recursive call and so each necessitates the creation an
additional evaluation context. The effect of these additional contexts is that each
annotation is applied to a new, ``blank'' context, so all the annotations are preserved. 
In the tail-recursive variant, there is no pending computation and therefore no additional
evaluation context. In this instance, the previous mark is overwritten with the new.

\subsection{Continuation-passing style}

The CPS transformation is a family of language transformations derived from Plotkin \cite{plotkin1975call} and designed to simplify  programs by representing all data and control flow uniformly and explicitly, in turn simplifying compiler construction and analyses such as optimization and verification \cite{sabry1994formal}. The standard variation adds a formal parameter to every function definition and an argument to every call site.

As an example, consider once again the two variants of the factorial function, sans continuation 
marks, given earlier. In CPS, the properly recursive variant can be expressed as
\begin{verbatim}
(define (fact n k) 
  (if (= n 0)
      (k 1)
      (fact (- n 1) (lambda (acc) (k (* n acc))))))
\end{verbatim}
and the tail-recursive variant as
\begin{verbatim}
(define (fact-tr n acc k)
  (if (= n 0)
      (k acc)
      (fact-tr (- n 1) (* n acc) k)))
\end{verbatim}. (For clarity, we have treated ``primitive'' functions--equality 
comparison, subtraction, and multiplication--in a direct manner. In contrast, a 
full CPS transformation would affect \emph{every} function.)

Notice that, in the first variation, each recursive call receives a newly-constructed 
$k$ encapsulating additional work to be performed at the completion of the recursive 
computation. In the second, $k$ is passed unmodified, so while computation occurs 
within each context, no \emph{additional} computation pends. From this example, we 
see that the CPS representation is ideal for understanding tail-call behavior as it 
is explicit that the continuation is preserved by the tail call.

The purpose of CPS does not lie solely in pedagogy, however. The reification of and 
consequent ability to directly manipulate the continuation is a powerful ability, 
analogous in power to the ability to \emph{capture} a continuation which some 
languages provide. In Scheme, this is accomplished with \emph{call/cc}, short for 
``call with current continuation''. This call takes one argument which itself is a 
function of one argument. \emph{call/cc} calls its argument, passing in a functional 
representation of the current continuation--the continuation present when \emph{call/cc} 
was invoked. This continuation function takes one argument which is treated as the result 
of \emph{call/cc} and runs this continuation to completion.

As a simple example,
\begin{verbatim}
(+ 1 (call/cc
       (lambda (k)
         (k 1))))
\end{verbatim}
returns $2$. In effect, invoking $k$ with the value 1 is the same as replacing 
the entire \emph{call/cc} invocation with the value 1.

Much of the power of \emph{call/cc} lies in the manifestation of the continuation 
as a function, giving it first-class status. It can be passed as an argument in 
function calls, invoked, and, amazingly, reinvoked at leisure. It is this 
reinvokeability that makes \emph{call/cc} the fundamental unit of control from 
which all other control structures can be built, including generators, coroutines, 
and threads.

%\emph{call/cc} is erroneously seen as incredibly heavyweight and overkill for 
%control (cite something). This conception probably comes from the conceptualization 
%of the continuation as the call stack and continuation capture as stack copy while 
%continuation call is stack installation. It is also seen as a form of \emph{goto} 
%which is known to obfuscate control flow and impede analysis (cite something). 
%[Can't and shouldn't talk much about the second. Probably take it out.] Bearing in 
%mind that the CPS transformation aids compilers, it is useful to investigate the 
%characterization of \emph{call/cc} within the standard CPS transformation.

In direct style, the definition of \emph{call/cc} is conceptually 
\begin{verbatim}
(define call/cc
  (lambda (f)
    (f (get-function-representing-continuation))))
\end{verbatim}
where \emph{get-function-representing-continuation} is an opaque function which 
leverages sweeping knowledge of the language implementation. The CPS definition 
is notably easier:
\begin{verbatim}
(define call/cc
  (lambda (f k)
    (f k k)))
\end{verbatim}

Variations on the standard CPS transformation can make the expression of certain 
control structures more straightforward. For instance, the ``double-barrelled'' 
CPS transformation is a variation wherein each function signature receives not 
one but two additional formal parameters, each a continuation. One application 
of this particular variation is error handling with one continuation argument 
representing the remainder of a successful computation and the other representing 
the failure contingency. It is especially useful in modelling exceptions and 
other non-local transfers of control in situations where the computation might fail.
In general, the nature of the CPS transformation allows it to untangle 
complicated, intricate control structures.

Similar transformations exist which express other programming language features 
such as security annotations \cite{wallach2000safkasi} and control structures 
such as procedures, exceptions, labelled jumps, coroutines, and 
backtracking. On top of other offerings, this places it in a category of tools 
to describe and analyze programming language features. (This category is also 
occupied by Moggi's computational lambda calculus--monads 
\cite{moggi1989computational}.)

\section{Project description}

We consider an extension of the call-by-value $\lambda$-calculus with 
facilities to manipulate continuation marks, introduced by Pettyjohn et al. 
\cite{pettyjohn2005continuations}, which we term $\lambda_{cm}$.

Figure \ref{language-grammar} presents the definition of $\lambda_{cm}$. 
Definitions of $E$ and $F$ signify computation contexts.The non-terminal 
$E$ is simply any term in the language but for the constraint that evaluation 
contexts interleave \emph{wcm} directives. The definition of $e$ establishes 
the forms of valid expressions in the language: applications, variables, 
values, \emph{wcm} forms, and \emph{ccm} forms. The definition of $v$ denotes 
that values in the language are $\lambda$-abstractions. (Also, notice that 
\emph{wcm} and \emph{ccm} have one parameter fewer than the forms introduced 
earlier. This is because $\lambda_{cm}$ expresses unkeyed marks. While it can 
be shown that a language with keyed marks can be expressed in terms of a 
language with unkeyed marks, such as $\lambda_{cm}$, this is not our concern.) 

\begin{figure}
\begin{align*}
E = &(\mathrm{wcm}\,v\,F) & e = &(e\,e)\\
    &F                    &     &x\\
F = &[]                   &     &v\\
    &(E\,e)               &     &(\mathrm{wcm}\,e\,e)\\
    &(v\,E)               &     &(\mathrm{ccm})\\
    &(\mathrm{wcm}\,E\,e) & v = & \lambda x. e
\end{align*}
\caption{$\lambda_{cm}$ syntax}
\label{language-grammar}
\end{figure}

Figure \ref{language-semantics} presents the semantics of $\lambda_{cm}$. The 
definitions therein establish the proper interpretation of various expressions. 
The first follows the typical definition of application. The second defines 
the tail behavior of the \emph{wcm} form. The third expresses that the 
\emph{wcm} form takes on the value of its body. Finally, the fourth defines 
the value of the \emph{ccm} form in terms of the $\chi$ metafunction which 
definition is given in figure \ref{chi-metafunction}.
The $\chi$ metafunction is defined over syntactic forms of the language, so its 
definition corresponds closely to the definition of $\lambda_{cm}$ found in 
figure \ref{language-grammar}.

\begin{figure}
\begin{align*}
E[(\lambda x.e)\,v]                         &\rightarrow E[e[x\leftarrow v]]\\
E[(\mathrm{wcm}\,v\,(\mathrm{wcm}\,v'\,e))] &\rightarrow E[(\mathrm{wcm}\,v'\,e)]\\
E[(\mathrm{wcm}\,v\,v')]                    &\rightarrow E[v']\\
E[(\mathrm{ccm})]                           &\rightarrow E[\chi(E)]
\end{align*}
\caption{$\lambda_{cm}$ evaluation rules}
\label{language-semantics}
\end{figure}

\begin{figure}
\begin{align*}
\chi([])                   &= \mathrm{empty}\\
\chi((E\,e))               &= \chi(E)\\
\chi((v\,E))               &= \chi(E)\\
\chi((\mathrm{wcm}\,E\,e)) &= \chi(E)\\
\chi((\mathrm{wcm}\,v\,E)) &= v : \chi(E)
\end{align*}
\caption{Definition of $\chi$ metafunction}
\label{chi-metafunction}
\end{figure}

We will present a transformation in the spirit of CPS from $\lambda_{cm}$ to 
the plain call-by-value $\lambda$-calculus. Furthermore, we will provide and 
prove a meaning-preservation theorem.

\subsection{Example CPS transform}

As an example of a CPS transform definition, we supply Fischer's transform 
(which we call $\mathcal{F}$) of the $\lambda$-calculus 
\cite{fischer1972lambda} below. Recall that terms in 
the $\lambda$-calculus take the form of lone variables $x$, 
$\lambda$-abstractions $\lambda x.M$, and applications 
$M\,N$ where $M$ and $N$ are themselves $\lambda$-calculus terms.

\begin{align*}
\mathcal{F}[x]           &= \lambda k.k\,x\\
\mathcal{F}[\lambda x.M] &= \lambda k.k(\lambda x.\mathcal{F}[M])\\
\mathcal{F}[M\,N]       &= \lambda k.\mathcal{F}[M](\lambda m.\mathcal{F}[N](\lambda n.(m\,n)\,k))
\end{align*}

The term binding induced by this transform allowed Plotkin to simulate the call-by-value dialect of the $\lambda$-calculus by the call-by-name and vice versa \cite{plotkin1975call}.

%go from compiling language with feature to language without feature -> 

%there is some paper by wadler and ?
%continuation marks as exceptions
%delimited continuations in CPS

%things to include:
%history of CPS
%compilers of continuation marks
%talk about call/cc as the quintessential use of CPS

\section{Validation}

An important property of our transformation is that it preserves meaning. If not, 
it would have few conceivable applications as programs in $\lambda_{cm}$ wouldn't 
correspond to programs in the $\lambda$-calculus in any meaningful sense.

This property is expressed by the following theorem:

\begin{cmtheorem}

For expressions $e$ and values $v$, $\mathcal{C}$ is a syntactic transformation with the 
property that
\[
e\rightarrow^{*}_{\lambda_{cm}}v\Rightarrow \mathcal{C}[e]\rightarrow^{*}_{\lambda}\mathcal{C}[v]
\]
where $\rightarrow_{\lambda_{cm}}$ signifies the reduction relation of $\lambda_{cm}$ 
and $\rightarrow_{\lambda}$ signifies the reduction relation of the $\lambda$-calculus.

\end{cmtheorem}

In prose, this theorem states that if an expression reduces to a value according to 
the $\lambda_{cm}$ semantics, the expression ``compiled'' to the $\lambda$-calculus 
reduces to the value ``compiled'' to the $\lambda$-calculus according to the 
$\lambda$-calculus semantics.

Proofs of this type are typically shown directly by induction on the reduction 
relation. We anticipate the main difficulty will be in ensuring proper tail-call 
behavior of adjacent marks. To aide testing and development, we will build a Redex 
\cite{findler2010redex} implementation of the transform.

\section{Thesis schedule}

We first intend to define the transform using Redex to obtain feedback. Once the 
transform has been defined and informally checked, we will proceed with formal 
verification returning to the first step as necessary. We expect the entire process 
to take less than five years.

%Abstract – 1 to 2 paragraphs summarizing the proposal. Introduction – 1 to 4 pages
%answering questions 1 and 2 above Related Work – 1 to 2 pages answering question 3 above.
%Thesis statement – 1 to 2 sentences stating what is to be demonstrated in your thesis.
%Project Description – 2 to 5 pages answering question 4 above. Validation – 1/2 to 2 pages
%answering question 5 above. Thesis Schedule – ¼ to ½ page specifying dates for completion
%of major milestones. Annotated Bibliography – 2 to 5 pages containing references for all
%work cited.

%%%%%%%%%%%%%%%%%%%%%%%%%

% Change these to reflect the bibliography style and bibtex database file you want to use
\bibliographystyle{annotate}
\bibliography{proposal}

\end{document}
