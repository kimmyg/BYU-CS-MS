\documentclass[ms]{byuprop}

% Options for this class include the following (* indicates default): 
%   10pt -- 10 point font size 
%   11pt -- 11 point font size 
%   12pt (*) -- 12 point font size 
%
%   ms -- produce a thesis proposal (off) 
%   areaexam -- produce a research area overview (off) 
%   phd -- produce a dissertation proposal (off) 
%
%   layout -- show layout lines on the pages, helps with overfull boxes (off) 
%   grid -- show a half-inch grid on every page, helps with printing (off)

% This command fixes my particular printer, which starts 0.03 inches too low, 
% shifting the whole page down by that amount. This shifts the document 
% content up so that it comes out right when printed. 
%
% Discovering this sort of behavior is best done by specifying the ``grid'' 
% option in the class parameters above.  It prints a 1/2 inch grid on every 
% page. You can then use a ruler to determine exactly what the printer is 
% doing. 
% Uncomment to shift content up (accounting for printer problems)
%\setlength{\voffset}{-.03in}

% Here we set things up for invisible hyperlinks in the document. This makes 
% the electronic version clickable without changing the way that the document 
% prints. It's useful, but optional. Note that if you use pdflatex, you 
% should change "ps2pdf" to "pdftex". 
\usepackage[
	ps2pdf,
	bookmarks=true,
	breaklinks=true,
	raiselinks=true,
	pdfborder={0 0 0},
	colorlinks=false,
	]{hyperref}

% Rewrite the itemize, description, and enumerate environments to have more 
% reasonable spacing: 
\newcommand{\ItemSep}{\itemsep 0pt}
\let\oldenum=\enumerate
\renewcommand{\enumerate}{\oldenum \ItemSep}
\let\olditem=\itemize
\renewcommand{\itemize}{\olditem \ItemSep}
\let\olddesc=\description
\renewcommand{\description}{\olddesc \ItemSep}

% Get a little less fussy about word spacing on a line.  Sometimes produces 
% ugly results, so keep your eyes peeled. 
\sloppy

% Important settings for the byuprop class. %
%%%%%%%%%%%%%%%%%%%%%%%%%%%%%%%%%%%%%%%%%%%%%

% Because I use these things in more than one place, I created new commands for 
% them. I did not use \providecommand because I absolutely want LaTeX to error 
% out if these already exist. 

\newcommand{\Title}{The Monadic Semantics of Continuation Marks}
\newcommand{\Author}{Kimball R. Germane} 
\newcommand{\SubmissionMonth}{April}
\newcommand{\SubmissionYear}{2012}

% Take these from the commands defined above
\title{\Title}
\author{\Author}
\monthsubmitted{\SubmissionMonth} 
\yearsubmitted{\SubmissionYear}

% Committee members 
\committeechair{Jay McCarthy} 
\committeemembera{} 
\committeememberb{}
\committeememberc{} 
\committeememberd{}

% Department graduate coordinator 
\graduatecoordinator{Dan Ventura}

%%%%%%%%%%%%%%%%%%%%%%%%%%%%%%%%%%%%%%%%%%%%%

% Set up the internal PDF information so that it becomes part of the document 
% metadata. The pdfinfo command will display this. Be sure to set the document 
% type and add your own keywords. 
\hypersetup{%
	pdftitle=\Title,%
	pdfauthor=\Author,%
	pdfsubject={Document Type, BYU CS Department: %
				Submitted \SubmissionMonth~\SubmissionYear, Created \today},%
	pdfkeywords={},%
}

% These packages allow the bibliography to be sorted alphabetically and allow references
% to more than one paper to be sorted and compressed (i.e. instead of [5,2,4,6] you get
% [2,4-6]) 

\usepackage[numbers,sort&compress]{natbib}
\usepackage{hypernat}

% Additional packages required for your specific thesis go here. I've left some I use as
% examples. 
%\usepackage{graphicx}
%\usepackage{pdfsync}

\begin{document}

% Produce the preamble
\maketitle


%\abstract{}

\section{Introduction}

% remember that the audience in this case actually does not know what these things are 
% so have a clear, useful exposition.

Thesis: Continuation marks are amenable to a monadic semantics.

% talk more about how they are useful here, elaboration will be below

There are certain tools that are indispensable to some programmers that concern the
behavior of their programs: debuggers, profilers, steppers, etc. Without these tools,
these programmers cannot justify the adoption of a language, however compelling it might
otherwise be. The nature of these tools typically requires them to be designed and
implemented in tandem with the language and privy to sometimes transient implementation
detail. Thus, they are a considerable burden on the language designer or implementor. The
combination of this dependence and burden can stall language development and proliferation.

Continuation marks \cite{clements2006portable} are a language feature that provides a
mechanism to annotate the dynamic context of a program. This feature allows the
association of arbitrary values with arbitrary keys for the dynamic extent of an execution
context and the inquiry of currently defined values of pending contexts for a given set of
keys. This is advantageous for programs that require dynamic information about a program
execution such as debuggers, profilers, and steppers because it allows them to be defined
at the same level as the language instead of some level below.

%For instance, <give a simple, useful, illustrative example here>.
% later we can talk about continuation marks and continuations, simulating dynamic scope, 
% recursion and tail recursion detection. 

% can I use the terms programming language feature and computational model interchangeably?

Monadic semantics originate from Moggi \cite{moggi1989computational} wherein they were
termed categorical semantics in reference to their category theoretic construction.
Monadic semantics describe the meaning of computational models such as partial
computability, side-effectfulness, and non-determinism. The appeal of using monads to
specify semantics is significant: it allows one to work with a pure computational logic,
such as the lambda calculus, with a particular model of computation in isolation,
insulating the interactions between different models. There are at least two direct
benefits to this: First, it helps one define and reason about a particular computational
model absent from the noise of other models and practical considerations. Second, no
language feature is an island and when it becomes necessary to investigate the
interactions between models, the monadic construction facilitates composition in a simple
way.

% talk about how they are recognized as a good way to do things.
% they allow reasoning about a feature or model in isolation but, surprisingly, also are 
% composable

Thus, the statement that continuation marks are amenable to a monadic semantics means
simply that we can characterize them with a monad, simultaneously allowing us to express
the essence of continuation marks and reason about its interaction with other
computational models.

% ... what does it actually mean? if we can show it satisfies the mono requirement, is it 
% a computational model?
% we can describe what they are at the core and also how they interact with other 
% computational models

\section{Related Work}

A denotational semantics of a language provides a precise definition of the meaning of
expressions in that language. However useful, the denotational approach is not modular in
that the addition of a language construct often requires global changes to the definition
\cite{liang2009modular}. For this reason, denotational semantics are often eschewed in
favor of a more tractable approach.

Behavioral semantics were developed in part to mitigate this problem.

% I need a good source to learn about behavioral semantics and operational semantics.
% also look into monadic denotational and monadic behavioral

Monadic semantics provide a truly modular approach to specifying language meaning and can
do so with precision.


A monad is an endofunctor in the category of monoids.

Moggi \cite{moggi1989computational} introduces monads as a categorical semantics of
computational models. He provides the following definitions:

Definition 2.1

A \emph{monad} over a category $\mathcal{C}$ is a triple $(T,\eta,\mu)$ where
$T:\mathcal{C}\rightarrow\mathcal{C}$, $\eta:Id_{\mathcal{C}}\rightarrow T$, and
$\mu:T^{2}\rightarrow T$ are \emph{natural transformations} and the following equations
hold:

\begin{enumerate}
\item $\mu_{TA};\mu_{A}=T(\mu_{A});\mu_{A}$
\item $\eta_{TA};\mu{A}=id_{TA}=T(\eta_{A});\mu_{A}$
\end{enumerate}

Defintion 2.2

A \emph{Kleisli triple} over $\mathcal{C}$ is a triple $(T,\eta,\_^{*})$, where
$T:Obj(\mathcal{C}\rightarrow Obj(\mathcal{C}$, $\eta_{A}:A\rightarrow TA$,
$f^{*}:TA\rightarrow TB$ for $f:A\rightarrow TB$ and the following equations hold:

\begin{enumerate}
\item $\eta_{A}^{*}=id_{TA}$
\item $eta_{A};f^{*}=f$
\item $f^{*};g^{*}=(f;g^{*})^{*}$
\end{enumerate}

These equations represent the monadic laws that all alleged monads must preserve. We will
appeal to them directly in the verification of our candidate monad.

\subsection{Monad Examples}

Some examples of monads might be illustrative. More, and in more depth, can be found in 
Moggi \cite{moggi1989computational}.

Example 1: Non-deterministic computations

\begin{itemize}
\item $T(\cdot)$ is the covariant powerset function, i.e., $T(A)=\mathcal{P}(A)$ and 
$T(f)(X)$ is the image of $X$ along $f$.
\item $\eta_{A}$ is the singleton map $a\mapsto \{a\}$
\item $\mu_{A}$ is the big union map $X\mapsto\bigcup X$
\end{itemize}

% explain how this is non-determinism and show how it satisfies the laws

non-termination monad
TA is A + {bottom}

nondeterminism monad
TA is P(A)

% use this as a jumping off point to talk about what monads are

Many tools require information about the dynamic context of a program: steppers,
debuggers, profilers, etc. Traditionally, these tools are developed at the same level as
the language implementation precisely because that implementation specifies the details
that enable these tools to function. This is problematic for at least two reasons. First,
it couples the implementation of the tool with the implementation of the language, which
increases the cost to port to other platforms. If users become dependent upon these tools,
it can stall the advancement of the language and the adoption of new language features.
Second and more critical, it makes these tools unsound. For instance, debuggers typically
examine programs which have been compiled without optimizations.

% is this correct and, if so, for what reason?

This means that, without costly equivalence guarantees (necessary?), the debugged program
is different than the deployed program. This is undesirable.

It is desirable to implement such tools at the same level as the language, removing
dependency upon the implementation an instead relying on definitional and behavioral
invariants. Continuation marks are a language-level feature that provide the information
necessary for these tools to function. Furthermore, languages which require stack
inspection to enforce security policies (\emph{Java}, \emph{C\#}) or support aspect
oriented programming (\emph{aspectj}) can be defined in terms of a simpler language with
continuation marks.

The feature of continuation marks itself is accessible via two surface level syntactic
forms: \emph{with-continuation-mark} and \emph{current-continuation-marks}.

\emph{with-continuation-mark} has three parameters: a key identifying the nature of the
continuation mark, an expression which value will be associated with this key, and an
expression within the dynamic extent of which a continuation mark will exist with this
key. In a Scheme-like syntax, it appears as so:

% use slatex
\begin{verbatim}
(with-continuation-mark /key-expression/ /value-expression/ /expression/)
\end{verbatim}

\emph{current-continuation-marks} has one parameter, a set of keys, and returns a list of
all the associated values set within the dynamic context of the invocation. Scheme-like,
it looks like this:

%look at Clement's dissertation to see the actual characterizations of these
\begin{verbatim}
(current-continuation-marks /key-set/)
\end{verbatim}

Importantly, the result of this invocation provides no evidence of any portion of the
dynamic context lacking continuation marks with the specified keys. This preserves the
ability to perform optimizations without exposing details which would render the
optimizations unsound. This also requires special consideration of a language that
supports tail call optimization which is not an optimization in the above sense since the
behavior of which is present in the semantic definition of the language. In fact, 
<<expression>> is in tail position; a language with tail call optimization will reflect 
this.

The canonical example to illustrate the behavior of continuation marks in the presence and
absence of tail recursion is the factorial function.

Figure \ref{fac-rec} illustrates the definitional recursive implementation (wc) of the
factorial function. In this [representation/expression], a cascade of multiplication
operations builds as the recursive calls are made. Each multiplication is computation that
must be performed after the recursive call of which the machine must keep track.


\begin{figure}
\label{fac-rec}
\begin{verbatim}
(define (fact n)
  (if (= n 0)
      1
      (* n (fact (- n 1))))
\end{verbatim}
\end{figure}

Figure \ref{fac-tail-rec} illustrates the tail recursive manifestation (wc?) of the
factorial function. In contrast to the function in figure \ref{fac-rec}, this formulation
performs the multiplication before the recursive call. Because no computation is performed
after the evaluation of the recursive call, that call is said to be in tail position.
Because the function has no pending computations, the execution context does not grow.

\begin{figure}
\label{fac-tail-rec}
\begin{verbatim}
(define (fact-tr n acc)
  (if (= n 0)
      acc
      (fact-tr (- n 1) (* n acc))))
\end{verbatim}
\end{figure}

% what this means for continuation marks in the face of such behavior. First, it may be
% useful to determine exactly to what the term \emph{continuation} refers. According to ...,
% a continuation is simply ``the rest of the computation''. Consider the following versions
% [wc] of the previous functions enhanced with continuation marks.

\begin{figure}
\label{fac-rec-cm}
\begin{verbatim}
(define (fact n)
  (if (= n 0)
      (begin
        (display (current-continuation-marks 'fact))
        1)
      (with-continuation-mark 'fact n (* n (fact (- n 1)))))
\end{verbatim}
\end{figure}

\begin{figure}
\label{fac-tail-rec-cm}
\begin{verbatim}
(define (fact-tr n acc)
  (if (= n 0)
      (begin
        (display (current-continuation-marks 'fact))
        acc)
      (with-continuation-mark 'fact n (fact-tr (- n 1) (* n acc))))
\end{verbatim}
\end{figure}

We would expect that, in the regular recursive variant, [the addition of new contexts
implies that each annotation is applied to a new context and so all the annotations are
preserved since each represents future computation to be performed.] In the tail-recursive
variant, there is no pending computation [and so we should expect that our previous
annotation be overwritten in some sense--replaced].


%explain continuation marks at a high level \cite{clements2006portable} %explain monadic
semantics at a high level

\section{Project Description}

produce and provide haskell implementation of a continuation mark monad produce and
provide haskell implementation of continuation marks according to formal semantics show
algebraically (i.e., solely through symbolic manipulation) that provided is a monad show
in the same way that what it models and what the semantics describe are equivalent profit

Haskell \cite{hudak1992report} is a lazy, purely functional programming language. As a
purely functional language, all side effects are explicit. This is a great boon when
reasoning about a function because distant, implicit effects are impossible in this
environment. In other words, a function will \emph{always} produce the same output given
the same input. %This property facilitates a bottom-up approach to program testing and
validation. all its e Because of this, functions can be analyzed singly and then composed
in a way which removes none of the guarantees attained in isolation. (We will see that
this is one of the useful characteristics of monads also.)

Another consequence of this is that input and output are not expressible as simple
functions. Each of these exhibit non-local effects and it is therefore impossible to
realize them as such. In order to perform a computation with I/O, we must model a
computation with I/O which is done by a properly crafted monad. The \texttt{IO} monad is
Haskell is the correct and endorsed method to perform I/O in a computation. Haskell has
many other regularly-used monads modelling computations with state, non-determinism,
continuations and exceptions, etc. Haskell is then a natural choice to implement and
verify a monad which models computations with continuation marks.

\section{Validation}

% it is intended that this section answer "how can we show this is a good solution?" this
% is the hardest of sciences and so it is implied that any proposed solution we provide will
% be accompanied by formal proof and we have stated we will supply it. that makes this
% section out-of-place.

In order to show that our construct is indeed a monad for continuation marks, we must show
that we our construct is a genuine monad and that it actually models continuation marks.

The mathematical basis of a monad lends three laws that fully specify the behavior of a 
monad. We will provide a proof that our construct satisfies these laws and can thus be 
correctly termed a monad. This proof is by nature algebraic.

Continuation marks themselves already have a formal operational semantics specified. To 
verify that our monad models continuation marks, we will prove that our monad is 
algebraically equivalent to the accepted operational semantic definition.

%It may be the case that we have something that \emph{appears} to be a continuation mark
monad but is not actually. In order to verify that our construct is a genuine continuation
mark monad, we must do the following things:

%First, we must show that it actually models continuation marks. Continuation marks
already have a denotational semantics defined. We can use equational
reasoning/algebraically compare the semantics of the monad against the denotation
semantics. What's more, if we express them in a suitable language, we can run our
research.

%Second, we must show that it actually is a monad. We can verify the construct satisfies
%the three monad laws algebraically.

%explain what is necessary to show that what we have are continuation marks and what we
%have is a monad

%monadic semantics accomplish [their purpose] by providing an abstraction barrier which
%hides irrelevant...

%something about real programs?

THESIS SCHEDULE

%Abstract – 1 to 2 paragraphs summarizing the proposal. Introduction – 1 to 4 pages
%answering questions 1 and 2 above Related Work – 1 to 2 pages answering question 3 above.
%Thesis statement – 1 to 2 sentences stating what is to be demonstrated in your thesis.
%Project Description – 2 to 5 pages answering question 4 above. Validation – 1/2 to 2 pages
%answering question 5 above. Thesis Schedule – ¼ to ½ page specifying dates for completion
%of major milestones. Annotated Bibliography – 2 to 5 pages containing references for all
%work cited.

%%%%%%%%%%%%%%%%%%%%%%%%%

% Change these to reflect the bibliography style and bibtex database file you want to use
\bibliographystyle{default}
\bibliography{proposal}

\end{document}
